% !TEX TS-program = pdflatex
% !TEX encoding = UTF-8 Unicode

% This is a simple template for a LaTeX document using the "article" class.
% See "book", "report", "letter" for other types of document.

\documentclass[11pt]{article} % use larger type; default would be 10pt

\usepackage[utf8]{inputenc} % set input encoding (not needed with XeLaTeX)
\usepackage[backend=biber, style=authoryear]{biblatex}

% basic bib file
\addbibresource{bibliography.bib}
% by lit type
\addbibresource{related_work.bib}
\addbibresource{network_lit.bib}
\addbibresource{general_cycling.bib}




%%% Examples of Article customizations
% These packages are optional, depending whether you want the features they provide.
% See the LaTeX Companion or other references for full information.

%%% PAGE DIMENSIONS
\usepackage{geometry} % to change the page dimensions
\geometry{a4paper} % or letterpaper (US) or a5paper or....
% \geometry{margin=2in} % for example, change the margins to 2 inches all round
% \geometry{landscape} % set up the page for landscape
%   read geometry.pdf for detailed page layout information

%\usepackage{graphicx} % support the \includegraphics command and options

\usepackage[parfill]{parskip} % Activate to begin paragraphs with an empty line rather than an indent

%%% PACKAGES
\usepackage{booktabs} % for much better looking tables
\usepackage{array} % for better arrays (eg matrices) in maths
\usepackage{paralist} % very flexible & customisable lists (eg. enumerate/itemize, etc.)
\usepackage{verbatim} % adds environment for commenting out blocks of text & for better verbatim
\usepackage{subfig} % make it possible to include more than one captioned figure/table in a single float
\usepackage{graphicx}
\usepackage{multirow} % allows multipel rows per cell
% These packages are all incorporated in the memoir class to one degree or another...

%%% HEADERS & FOOTERS
\usepackage{fancyhdr} % This should be set AFTER setting up the page geometry
\pagestyle{fancy} % options: empty , plain , fancy
\renewcommand{\headrulewidth}{0pt} % customise the layout...
\lhead{}\chead{}\rhead{}
\lfoot{}\cfoot{\thepage}\rfoot{}

%%% SECTION TITLE APPEARANCE
\usepackage{sectsty}
\allsectionsfont{\sffamily\mdseries\upshape} % (See the fntguide.pdf for font help)
% (This matches ConTeXt defaults)

%%% ToC (table of contents) APPEARANCE
\usepackage[nottoc,notlof,notlot]{tocbibind} % Put the bibliography in the ToC
\usepackage[titles,subfigure]{tocloft} % Alter the style of the Table of Contents
\renewcommand{\cftsecfont}{\rmfamily\mdseries\upshape}
\renewcommand{\cftsecpagefont}{\rmfamily\mdseries\upshape} % No bold!

%%% END Article customizations

%%% The "real" document content comes below...

\title{\vspace{-3.0cm}Review of Literature on Cycle Network Analysis}
\author{Author}
%\date{} % Activate to display a given date or no date (if empty),
         % otherwise the current date is printed 

\begin{document}
\maketitle


\section{Introduction}

Half of the world's population lives in cities. Cycling is of obvious importance to the well being of the world's  urban population in terms of the macro-climate crisis, local pollution problems, and public health issues. A commonly researched question is ``what factors have the most significant effect on cycling rates?'' The review of literature that follows will argue that the majority of work on this question approaches the matter from a perspective of discrete policy intervention and the effect of marginal improvements on cycling. This fails to build a comprehensive theory of cycling rates that a network analysis approach can offer. Only network theory can offer a high level look at the level of service in a community. The review that follows takes the structure of an initial look at typical literature considering how to promote cycling, recent attempts to use network analysis to accomplish this same goal, and closes with a look at work in network analysis that can inform applying the field to cycle networks. The overarching theme of the review and the research that is proposed is the idea that no individual intervention is sufficient to change behavior until a comprehensive network of infrastructure that reflects the importance of locations according to centrality is created. The implication of this view is that infrastructure changes are unlikely to have a meaningful effect on behavior in the system until a critical mass or tipping point is reached. 

\section{Relevant Literature regarding cycling generally}




\section{Literature addressing cycling From a network perspective}

Dill, 4 types of cyclists, ``strong and fearless'', ``enthused and confident'', ``interested but concerned'', no way no how''.   \cite{dill2013four}

Ellett, State Paranoia and urban Cycling: fear is a significant factor during an urban cycling trip. \cite{ellett2018state}

Fajans, Why Bicyclists Hate Stop Signs, built environment has a meaingful impact on the energy required for a journey by bicycle. \cite{fajans2001bicyclists}

Gosse, Estimating Spatially and Temporally continous bicycle columes by using sparse data. Difficult to get an estimate of bike traffic with the data available. \cite{gosse2014estimating}

Gu, The cost effectiveness of bike lanes in New York City, ROI is high for bike lanes. \cite{gu2017cost}

Heinen, Commuting by bicycle, an overveiw of the lit: investigates the determinants for commuting by bicycle. \cite{heinen2010commuting}

Kondo: Where do bike lanes work best a Bayesian spatial model of bicycle lanes and bicycle crashes. Bike lanes have different values in different places. \cite{kondo2018bike}

Li, Physical environments influencing bicyclists' perception of comfort on separated and on-street bicycle facilities. Simple look at perceived safety in different cycling environments. \cite{li2012physical}

Parkin, Models of perceived cycling risk and route acceptability. The factors that influence perceived safety \cite{parkin2007models}

Savan, Integrated Strategies to accelerate the adoption of cycling for transportation, Psychological behavioral change approach to increasing cycling. \cite{savan2017integrated}

Stinson, Route preferences of expereinced and inexperienced bicycle commuters. Different types of cyclists react differently to traffic stress. \cite{stinson2005comparison}

Thigpen, States of change approach to explore opportunities for increased bicycle commuting. Focuses on changing cyclists rather than infrastructure. \cite{thigpen2015using}

Vandenbulcke, Predicting cycling accident risk in Brussels: A Spatial case control approach. The things cyclists perceive as dangerous are dangerous. \cite{vandenbulcke2014predicting}













\section{Network analysis Literature relevant to the research methodology proposed}

\section{Intended research procedure}

\section{}

\section{}

%%%%%%%%%%%%%%%%%%%%%%%%%%%%%%%%%%%%%%%%%%%%%%%%%%%%%%%%%%%%%%%%%%%%%%%%%%%%%%%%%%%%%%%%%%%%%%%%%%%%%%%%%%%%



\section{Introduction}

	Cycling has been promoted with increasing frequency as a solution to a number of problems facing cities and the world currently including the climate crisis, local pollution, and population health problems. The following review of literature will consider previous work on the effect of various changes to cycling infrastructure on the population's propensity to cycle. This will indicate the potential for network analysis to improve the understanding of the relationship between cycling infrastructure and behavior. Then the review will turn to a series of work that provides a framework for applying network analysis to cycling infrastructure in order to evaluate the resilience and efficiency of the infrastructure and predict the effect of a change. 
	
\section{The Effect of Infrastructure Changes on Cycling Rates}

	Buehler \& Dill's \cite{Buehler and Dill} particularly helpful review of literature on bike networks uses a framework that groups the literature by two categories, that focusing on the links of the network, different types of cycle lanes, and that focusing on the nodes of the network, intersections. It ends by addressing the possibility of a cohesive study of a bicycle network holistically. 
	
	For instance \textit{One study linked cycling levels to self reported measures of the bicycling environment, including being able to take shortcuts on a bicycle compared to routes available to cars. This measure of connectivity was found to be significant (Titze, Stronegger, Janschitz, \& Oja, 2008).}
	
	\textit{ a recent paper using aggregate bicycle commuting data from 74 US cities (Schoner \& Levinson, 2014). The authors develop several different measures that represent the size, connectivity, density, fragmentation, and directness of the bicycle network. The density of the bikeway network (all types of facilities combined) had the largest elasticity value, larger than connectivity, fragmentation, and directness combined.}
	
	This includes a Bicycle Compatibility Index (BCI) combination of distance and safety, Bicycle level of service (BLOS) from ``Highway Capacity Manual'', \cite{Lowry} and Level of Traffic Stress (LTS) \cite{Mekuria and Furth} a four point scale based on architectural features of the route.
	
	Basically, these four don't really constitute a holistic approach still. 
	
	Difficulty of validating the effect because there isn't a lot of cyclist behavior data available.
	
	
Since Buehler's literature review was published 115 subsequent papers have cited it but only a few have taken the author's central recommendation that a holistic approach to evaluating a bike network be employed. 
	
	
	
\section{Crucial Authors}



\begin{itemize}
\item Buehler \cite{Buehler2016}

Lit Review of cycling network research

\item Furth Low Stress Bicycling Network Connectivity

\item 
  \begin{itemize}
  \item one point one
  \item one point two
  \end{itemize}
\end{itemize}

Buehler

Furth
Lowry
	
	
	
\section{Research Framework}

Road network
	attach attributes to links and nodes
		bike lane with type
		traffic characteristics
		road characteristics
		traffic incidents
		
Remove links and nodes according to attributes

Compare distribution of centrality between estimated bike network and real road network. 

Use Cycle Hire data to validate centrality preference?
	But hires are station to station so would have to show that the probability of a station being a destination for a trip is influenced by stations centrality. 


	
	









\section{Introduction}

	More than half of the world's population lives in cities. Current mode share of transportation is problematic on a global scale, where the climate emergency is fueled by carbon emmissions, a significant portion of which are driven by transportation, and on a local scale, where particulate and Nitrous Oxide pose a public health threat. Thus changing transportation methods is a clear necessity in addressing one of the world's most crucial problems. 
	
	Part of this change can come from increasing bicycle usage in a number of travel types. The choice to ride a bike is a complex one, involving fitness, origin and destination, safety and security infrastructure for cyclists, and weather. A less considered factor is that of network strength and resilience.
	
	Network analysis offers a toolset for assessing the extent to which the road and cycle-lane network is a hindrance to a person's preference for cycling. Building a cycle network efficiently is a unique problem in network analysis to some extent because the challenge is to build the most efficient network on top of the existing road network with the lowest cost to metrics for the existing network. 
	
	Thus this work aims to answer the question, how efficiently has the cycle infrastructure added to London's streets been implemented? To do this, a framework for building an efficient network across an existing network must be defined. This requires defining a cost and benefit of adding a cycle lane to a road/link in the network. This should consider the centrality of the link, and the change in accessibility to car and bike uses as a result of the change.
	
	Parts
	
how to define the cost and benefit of this stuff?

How to define the optimal bike network?

How to 




Steps:

	get london road network
		just the data
		get the distribution of centrality measures
	get the london bike network
		just the data
		get the distribution of centrality measures
		
	give a road an index grade
		using accident stats and road characteristics
		
	add ``safe'' roads to cycle network
	
	compare disributions of centrality scores between road network and bike network. 
	
	compare network before and after Super Highway constructions
	
	
	
	
	
Lit Review Structure

	Research goal  and specific question
	
		Goal is to figure out how to apply network science to the study of bike infrastructure. Specifically, analyze the resilience and efficiency of a cycle network. 
	
	Context
	
		Global climate crisis, 
		local urban pollution
		health crisis

	Why it matters
		
		Cycling solves too many problems to be ignored
		Most academics that might be interested in cycling don't have a background in network analysis or the required skillset. 
	
	Previous work on cycling
	
		Cite Buehler Lit review
		Different environmental and demographic predictors of cycling
		Measurement of ``effect'' of different infrastructure stuff
		civil engineering type of work on specific situations
	
	Previous attempts to use network science for biking stuff
		Furth Level of traffic stress thing
			I think he just interprets the result qualitatively
	
	Relevant network analysis papers
		
		Planar graphs
		Connection between centrality and economic activity
			Porta 2012, Barcelona
		Transitions in spatial networks: percolation and small world stuff
		Multiplex networks probably not relevant because it's hard to take a bike on a train or bus or whatever
		Urban accessibility measurement
			Porta 
			Biazzo
			Need to weight edges but inverse of distance
	
	Methodology
	
	

Possible Research Questions
	Conservative: What does London look like to a cyclist? Build network according to different rules and compare
		traffic stress
			intersections
			streets
		no right turns
		only official bike network
		
		
	Moderate: community detection using the stuff from the conservative project, where is the network the strongest?
	Aggressive: how closely has the London network followed an optimal growth of network on network? 
		This could use standard measures of street centrality for the road network and look at accessibility for the most important streets. 
		Or compare centrality distribution of bike networks to streets, use street centrality for given bike lane, are lanes relatively central or no? 
		Or compare evolution of rd networks to evolution of bike networks
		
Straightness is an interesting one but distance matters most I guess. 
Congestion is not really 


	
	


\begin{itemize}
\item one
\item two
  \begin{itemize}
  \item one point one
  \item one point two
  \end{itemize}
\end{itemize}

\subsection{Subsection 1}

\begin{figure}
\centering
\includegraphics[width=0.8\textwidth]{example}
\caption{Correlation between station/node metrics}
\end{figure}

\begin{tabular}{|l|l|l|l|}\hline
  \multirow{10}{*}{numeric literals} 				& \multirow{5}{*}{integers} 	& in decimal 					& \verb|8743| \\ \cline{3-4}
  					    				& 				       	& \multirow{2}{*}{in octal}   		& \verb|0o7464| \\ \cline{4-4}
  					    				& 					& 						& \verb|0O103| \\ \cline{3-4}
  					    				& 					& \multirow{2}{*}{in hexadecimal}	& \verb|0x5A0FF| \\ \cline{4-4}
 				 	    				& 					& 						& \verb|0xE0F2| \\ \cline{2-4}
  					    				& \multirow{5}{*}{fractionals} 	& \multirow{5}{*}{in decimal} 		& \verb|140.58| \\ \cline{4-4}
 				 					& 					& 						& \verb|8.04e7| \\ \cline{4-4}
  									& 					& 						& \verb|0.347E+12| \\ \cline{4-4}
  									& 					& 						& \verb|5.47E-12| \\ \cline{4-4}
  									& 					& 						& \verb|47e22| \\ \cline{1-4}
  \multicolumn{3}{|l|}{\multirow{3}{*}{char literals}} 													& \verb|'H'| \\ \cline{4-4}
  \multicolumn{3}{|l|}{} 																	& \verb|'\n'| \\ \cline{4-4}          %% here
  \multicolumn{3}{|l|}{} 																	& \verb|'\x65'| \\ \cline{1-4}        %% here
  \multicolumn{3}{|l|}{\multirow{2}{*}{string literals}} 												& \verb|"bom dia"| \\ \cline{4-4}
  \multicolumn{3}{|l|}{} 																	& \verb|"ouro preto\nmg"| \\ \cline{1-4}          %% here
\end{tabular}




\begin{verbatim}
use pseudocode
\end{verbatim}

\textit{italics}
\textbf{bold}

XXXX words excluding headings, figures, and references. \\

\nocite{*}

\medskip


\printbibliography


\end{document}
