% !TEX TS-program = pdflatex
% !TEX encoding = UTF-8 Unicode

% This is a simple template for a LaTeX document using the "article" class.
% See "book", "report", "letter" for other types of document.

\documentclass[11pt]{article} % use larger type; default would be 10pt

\usepackage[utf8]{inputenc} % set input encoding (not needed with XeLaTeX)
\usepackage[backend=biber, style=authoryear]{biblatex}
\addbibresource{bibliography.bib}
%%% Examples of Article customizations
% These packages are optional, depending whether you want the features they provide.
% See the LaTeX Companion or other references for full information.

%%% PAGE DIMENSIONS
\usepackage{geometry} % to change the page dimensions
\geometry{a4paper} % or letterpaper (US) or a5paper or....
% \geometry{margin=2in} % for example, change the margins to 2 inches all round
% \geometry{landscape} % set up the page for landscape
%   read geometry.pdf for detailed page layout information

%\usepackage{graphicx} % support the \includegraphics command and options

\usepackage[parfill]{parskip} % Activate to begin paragraphs with an empty line rather than an indent

%%% PACKAGES
\usepackage{booktabs} % for much better looking tables
\usepackage{array} % for better arrays (eg matrices) in maths
\usepackage{paralist} % very flexible & customisable lists (eg. enumerate/itemize, etc.)
\usepackage{verbatim} % adds environment for commenting out blocks of text & for better verbatim
\usepackage{subfig} % make it possible to include more than one captioned figure/table in a single float
\usepackage{graphicx}
\usepackage{multirow} % allows multipel rows per cell
% These packages are all incorporated in the memoir class to one degree or another...

%%% HEADERS & FOOTERS
\usepackage{fancyhdr} % This should be set AFTER setting up the page geometry
\pagestyle{fancy} % options: empty , plain , fancy
\renewcommand{\headrulewidth}{0pt} % customise the layout...
\lhead{}\chead{}\rhead{}
\lfoot{}\cfoot{\thepage}\rfoot{}

%%% SECTION TITLE APPEARANCE
\usepackage{sectsty}
\allsectionsfont{\sffamily\mdseries\upshape} % (See the fntguide.pdf for font help)
% (This matches ConTeXt defaults)

%%% ToC (table of contents) APPEARANCE
\usepackage[nottoc,notlof,notlot]{tocbibind} % Put the bibliography in the ToC
\usepackage[titles,subfigure]{tocloft} % Alter the style of the Table of Contents
\renewcommand{\cftsecfont}{\rmfamily\mdseries\upshape}
\renewcommand{\cftsecpagefont}{\rmfamily\mdseries\upshape} % No bold!

%%% END Article customizations

%%% The "real" document content comes below...

\title{\vspace{-3.0cm}Title}
\author{Author}
%\date{} % Activate to display a given date or no date (if empty),
         % otherwise the current date is printed 

\begin{document}
\maketitle

\section{Introduction}

	More than half of the world's population lives in cities. Current mode share of transportation is problematic on a global scale, where the climate emergency is fueled by carbon emmissions, a significant portion of which are driven by transportation, and on a local scale, where particulate and Nitrous Oxide pose a public health threat. Thus changing transportation methods is a clear necessity in addressing one of the world's most crucial problems. 
	
	Part of this change can come from increasing bicycle usage in a number of travel types. The choice to ride a bike is a complex one, involving fitness, origin and destination, safety and security infrastructure for cyclists, and weather. A less considered factor is that of network strength and resilience.
	
	Network analysis offers a toolset for assessing the extent to which the road and cycle-lane network is a hindrance to a person's preference for cycling. Building a cycle network efficiently is a unique problem in network analysis to some extent because the challenge is to build the most efficient network on top of the existing road network with the lowest cost to metrics for the existing network. 
	
	Thus this work aims to answer the question, how efficiently has the cycle infrastructure added to London's streets been implemented? To do this, a framework for building an efficient network across an existing network must be defined. This requires defining a cost and benefit of adding a cycle lane to a road/link in the network. This should consider the centrality of the link, and the change in accessibility to car and bike uses as a result of the change.
	
	Parts
	
how to define the cost and benefit of this stuff?

How to define the optimal bike network?

How to 




Steps:

	get london road network
		just the data
		get the distribution of centrality measures
	get the london bike network
		just the data
		get the distribution of centrality measures
		
	give a road an index grade
		using accident stats and road characteristics
		
	add ``safe'' roads to cycle network
	
	compare disributions of centrality scores between road network and bike network. 
	
	compare network before and after Super Highway constructions
	
	
	
	
	
Lit Review Structure

	Research goal  and specific question
	
		Goal is to figure out how to apply network science to the study of bike infrastructure. Specifically, analyze the resilience and efficiency of a cycle network. 
	
	Context
	
		Global climate crisis, 
		local urban pollution
		health crisis

	Why it matters
		
		Cycling solves too many problems to be ignored
		Most academics that might be interested in cycling don't have a background in network analysis or the required skillset. 
	
	Previous work on cycling
	
		Cite Buehler Lit review
		Different environmental and demographic predictors of cycling
		Measurement of ``effect'' of different infrastructure stuff
		civil engineering type of work on specific situations
	
	Previous attempts to use network science for biking stuff
		Furth Level of traffic stress thing
			I think he just interprets the result qualitatively
	
	Relevant network analysis papers
		
		Planar graphs
		Connection between centrality and economic activity
			Porta 2012, Barcelona
		Transitions in spatial networks: percolation and small world stuff
		Multiplex networks probably not relevant because it's hard to take a bike on a train or bus or whatever
		Urban accessibility measurement
			Porta 
			Biazzo
			Need to weight edges but inverse of distance
	
	Methodology
	
	

Possible Research Questions
	Conservative: What does London look like to a cyclist? Build network according to different rules and compare
		traffic stress
			intersections
			streets
		no right turns
		only official bike network
		
		
	Moderate: community detection using the stuff from the conservative project, where is the network the strongest?
	Aggressive: how closely has the London network followed an optimal growth of network on network? 
		This could use standard measures of street centrality for the road network and look at accessibility for the most important streets. 
		Or compare centrality distribution of bike networks to streets, use street centrality for given bike lane, are lanes relatively central or no? 
		Or compare evolution of rd networks to evolution of bike networks
		
Straightness is an interesting one but distance matters most I guess. 
Congestion is not really 


	
	


\begin{itemize}
\item one
\item two
  \begin{itemize}
  \item one point one
  \item one point two
  \end{itemize}
\end{itemize}

\subsection{Subsection 1}

\begin{figure}
\centering
\includegraphics[width=0.8\textwidth]{example}
\caption{Correlation between station/node metrics}
\end{figure}

\begin{tabular}{|l|l|l|l|}\hline
  \multirow{10}{*}{numeric literals} 				& \multirow{5}{*}{integers} 	& in decimal 					& \verb|8743| \\ \cline{3-4}
  					    				& 				       	& \multirow{2}{*}{in octal}   		& \verb|0o7464| \\ \cline{4-4}
  					    				& 					& 						& \verb|0O103| \\ \cline{3-4}
  					    				& 					& \multirow{2}{*}{in hexadecimal}	& \verb|0x5A0FF| \\ \cline{4-4}
 				 	    				& 					& 						& \verb|0xE0F2| \\ \cline{2-4}
  					    				& \multirow{5}{*}{fractionals} 	& \multirow{5}{*}{in decimal} 		& \verb|140.58| \\ \cline{4-4}
 				 					& 					& 						& \verb|8.04e7| \\ \cline{4-4}
  									& 					& 						& \verb|0.347E+12| \\ \cline{4-4}
  									& 					& 						& \verb|5.47E-12| \\ \cline{4-4}
  									& 					& 						& \verb|47e22| \\ \cline{1-4}
  \multicolumn{3}{|l|}{\multirow{3}{*}{char literals}} 													& \verb|'H'| \\ \cline{4-4}
  \multicolumn{3}{|l|}{} 																	& \verb|'\n'| \\ \cline{4-4}          %% here
  \multicolumn{3}{|l|}{} 																	& \verb|'\x65'| \\ \cline{1-4}        %% here
  \multicolumn{3}{|l|}{\multirow{2}{*}{string literals}} 												& \verb|"bom dia"| \\ \cline{4-4}
  \multicolumn{3}{|l|}{} 																	& \verb|"ouro preto\nmg"| \\ \cline{1-4}          %% here
\end{tabular}




\begin{verbatim}
use pseudocode
\end{verbatim}

\textit{italics}
\textbf{bold}

XXXX words excluding headings, figures, and references. \\

\nocite{*}

\medskip


\printbibliography


\end{document}
