% !TEX TS-program = pdflatex
% !TEX encoding = UTF-8 Unicode

% This is a simple template for a LaTeX document using the "article" class.
% See "book", "report", "letter" for other types of document.

\documentclass[11pt]{article} % use larger type; default would be 10pt

\usepackage[utf8]{inputenc} % set input encoding (not needed with XeLaTeX)
\usepackage[backend=biber, style=authoryear]{biblatex}
\addbibresource{bibliography.bib}
%%% Examples of Article customizations
% These packages are optional, depending whether you want the features they provide.
% See the LaTeX Companion or other references for full information.

%%% PAGE DIMENSIONS
\usepackage{geometry} % to change the page dimensions
\geometry{a4paper} % or letterpaper (US) or a5paper or....
% \geometry{margin=2in} % for example, change the margins to 2 inches all round
% \geometry{landscape} % set up the page for landscape
%   read geometry.pdf for detailed page layout information

%\usepackage{graphicx} % support the \includegraphics command and options

\usepackage[parfill]{parskip} % Activate to begin paragraphs with an empty line rather than an indent

%%% PACKAGES
\usepackage{booktabs} % for much better looking tables
\usepackage{array} % for better arrays (eg matrices) in maths
\usepackage{paralist} % very flexible & customisable lists (eg. enumerate/itemize, etc.)
\usepackage{verbatim} % adds environment for commenting out blocks of text & for better verbatim
\usepackage{subfig} % make it possible to include more than one captioned figure/table in a single float
\usepackage{graphicx}
\usepackage{multirow} % allows multipel rows per cell
% These packages are all incorporated in the memoir class to one degree or another...

%%% HEADERS & FOOTERS
\usepackage{fancyhdr} % This should be set AFTER setting up the page geometry
\pagestyle{fancy} % options: empty , plain , fancy
\renewcommand{\headrulewidth}{0pt} % customise the layout...
\lhead{}\chead{}\rhead{}
\lfoot{}\cfoot{\thepage}\rfoot{}

%%% SECTION TITLE APPEARANCE
\usepackage{sectsty}
\allsectionsfont{\sffamily\mdseries\upshape} % (See the fntguide.pdf for font help)
% (This matches ConTeXt defaults)

%%% ToC (table of contents) APPEARANCE
\usepackage[nottoc,notlof,notlot]{tocbibind} % Put the bibliography in the ToC
\usepackage[titles,subfigure]{tocloft} % Alter the style of the Table of Contents
\renewcommand{\cftsecfont}{\rmfamily\mdseries\upshape}
\renewcommand{\cftsecpagefont}{\rmfamily\mdseries\upshape} % No bold!

%%% END Article customizations

%%% The "real" document content comes below...

\title{\vspace{-3.0cm}Title}
\author{Hugh Kelley}
\date{} % Activate to display a given date or no date (if empty),
         % otherwise the current date is printed 

\begin{document}
\maketitle

\section{Abstract}

\section{List of Tables}

\section{List of Figures}

% TOC

%%%%%%%%%%%%%%%%%%%%%%%%%%%%%%%%%%%%%%%%%%%%%%%%%%%%%%%%%%%%%%%%%%%%%%%%%%%%%%%%%%%%%%%%%%%%%%%%%%%%%%%%%%%%%%%%%
\section{Research Goal and Overview}
What makes a city for cycling? The conditions are clear when one sees them, but defining the characteristics quantitatively is challenging. 

This research is intended to build a cohesive methodology for defining the overall suitability of a city to transport by bicycle. It extends the literature in three ways; extending quantitative research to a holistic approach rather than focusing on individual edges or nodes, extending holistic approaches to quantitative outcomes rather than visual or anecdotal conclusions, and by emphasizing the use of open source tools and data that are available to any member of community. 

The research provides a methodology that allows a community to identify nodes and edges for improvement that will have the largest impact on the overall network, quantify the expected improvement, and thus require from elected officials specific changes to the roads in their communities. 

\subsection{ethical risks}

This project relies entirely on publicly accessible data. 

%%%%%%%%%%%%%%%%%%%%%%%%%%%%%%%%%%%%%%%%%%%%%%%%%%%%%%%%%%%%%%%%%%%%%%%%%%%%%%%%%%%%%%%%%%%%%%%%%%%%%%%%%%%%%%%%%
\section{Introduction}


\subsection{Cycling mode share}

\subsection{The decision to cycle}

\subsection{Route Choice}

\subsection{The role of the researcher}

%%%%%%%%%%%%%%%%%%%%%%%%%%%%%%%%%%%%%%%%%%%%%%%%%%%%%%%%%%%%%%%%%%%%%%%%%%%%%%%%%%%%%%%%%%%%%%%%%%%%%%%%%%%%%%%%%
\section{Literature}

\subsection{The decision to cycle}

\subsection{Cycling Infrastructure}

\subsection{Cycling Danger}

\subsection{Cycling Networks}

\subsection{Network Analysis and Accessibility}

%%%%%%%%%%%%%%%%%%%%%%%%%%%%%%%%%%%%%%%%%%%%%%%%%%%%%%%%%%%%%%%%%%%%%%%%%%%%%%%%%%%%%%%%%%%%%%%%%%%%%%%%%%%%%%%%%
% Methods and Data need a good structure because they rely on each other
\section{Methods}

\subsection{Data Sources}

\subsubsection{Open Street Map}

\subsubsection{2011 Census Journey to work data}

\subsubsection{LSOA boundaries}

\subsubsection{Road KSI data}

\subsection{Data import, storage, cleaning, and joining}

\subsection{Defining Cycle Networks}

How were the representative networks built? 

\subsection{Travel Times}

\subsection{Street Type and Street Danger}

\subsection{Accessibility}

%%%%%%%%%%%%%%%%%%%%%%%%%%%%%%%%%%%%%%%%%%%%%%%%%%%%%%%%%%%%%%%%%%%%%%%%%%%%%%%%%%%%%%%%%%%%%%%%%%%%%%%%%%%%%%%%%
\section{Analysis \& Results}

\subsection{Network Characteristics}

\subsection{Travel Times}

\subsection{Cycling Danger}

\subsection{Accessibility}

%%%%%%%%%%%%%%%%%%%%%%%%%%%%%%%%%%%%%%%%%%%%%%%%%%%%%%%%%%%%%%%%%%%%%%%%%%%%%%%%%%%%%%%%%%%%%%%%%%%%%%%%%%%%%%%%%
\sections{Conclusions}

\subsection{Results}

\subsection{Limitations}

\subsection{Opportunities for improvement and extension}




%%%%%%%%%%%%%%%%%%%%%%%%%%%%%%%%%%%%%%%%%%%%%%%%%%%%%%%%%%%%%%%%%%%%%%%%%%%%%%%%%%%%%%%%%%%%%%%%%%%%%%%%%%%%%%%%%
\sections{Appendix}




\begin{itemize}
\item one
\item two
  \begin{itemize}
  \item one point one
  \item one point two
  \end{itemize}
\end{itemize}

\subsection{Subsection 1}

\begin{figure}
\centering
\includegraphics[width=0.8\textwidth]{example}
\caption{Correlation between station/node metrics}
\end{figure}

\begin{tabular}{|l|l|l|l|}\hline
  \multirow{10}{*}{numeric literals} 				& \multirow{5}{*}{integers} 	& in decimal 					& \verb|8743| \\ \cline{3-4}
  					    				& 				       	& \multirow{2}{*}{in octal}   		& \verb|0o7464| \\ \cline{4-4}
  					    				& 					& 						& \verb|0O103| \\ \cline{3-4}
  					    				& 					& \multirow{2}{*}{in hexadecimal}	& \verb|0x5A0FF| \\ \cline{4-4}
 				 	    				& 					& 						& \verb|0xE0F2| \\ \cline{2-4}
  					    				& \multirow{5}{*}{fractionals} 	& \multirow{5}{*}{in decimal} 		& \verb|140.58| \\ \cline{4-4}
 				 					& 					& 						& \verb|8.04e7| \\ \cline{4-4}
  									& 					& 						& \verb|0.347E+12| \\ \cline{4-4}
  									& 					& 						& \verb|5.47E-12| \\ \cline{4-4}
  									& 					& 						& \verb|47e22| \\ \cline{1-4}
  \multicolumn{3}{|l|}{\multirow{3}{*}{char literals}} 													& \verb|'H'| \\ \cline{4-4}
  \multicolumn{3}{|l|}{} 																	& \verb|'\n'| \\ \cline{4-4}          %% here
  \multicolumn{3}{|l|}{} 																	& \verb|'\x65'| \\ \cline{1-4}        %% here
  \multicolumn{3}{|l|}{\multirow{2}{*}{string literals}} 												& \verb|"bom dia"| \\ \cline{4-4}
  \multicolumn{3}{|l|}{} 																	& \verb|"ouro preto\nmg"| \\ \cline{1-4}          %% here
\end{tabular}




\begin{verbatim}
use pseudocode
\end{verbatim}

\textit{italics}
\textbf{bold}

XXXX words excluding headings, figures, and references. \\

\nocite{*}

\medskip


\printbibliography


\end{document}
