% !TEX TS-program = pdflatex
% !TEX encoding = UTF-8 Unicode

% for a word count:
%https://app.uio.no/ifi/texcount/online.php

% This is a simple template for a LaTeX document using the "article" class.
% See "book", "report", "letter" for other types of document.

\documentclass[11pt]{article} % use larger type; default would be 10pt


\usepackage[utf8]{inputenc} % set input encoding (not needed with XeLaTeX)
\usepackage[backend=biber, style=authoryear]{biblatex}

% basic bib file
%\addbibresource{bibliography.bib}
% by lit type
\addbibresource{../bib/software.bib}
\addbibresource{../Literature_Review/Literature/Past_Similar_Work/related_work.bib}
\addbibresource{../Literature_Review/Literature/Network_Analysis/network_lit.bib}
\addbibresource{../Literature_Review/Literature/Bicycling/general_cycling.bib}


%%% Examples of Article customizations
% These packages are optional, depending whether you want the features they provide.
% See the LaTeX Companion or other references for full information.

%%% PAGE DIMENSIONS
\usepackage{geometry} % to change the page dimensions
\geometry{a4paper} % or letterpaper (US) or a5paper or....
% \geometry{margin=2in} % for example, change the margins to 2 inches all round
% \geometry{landscape} % set up the page for landscape
%   read geometry.pdf for detailed page layout information

%\usepackage{graphicx} % support the \includegraphics command and options

\usepackage[parfill]{parskip} % Activate to begin paragraphs with an empty line rather than an indent

%%% PACKAGES
\usepackage{hyperref}
\usepackage{booktabs} % for much better looking tables
\usepackage{array} % for better arrays (eg matrices) in maths
\usepackage{paralist} % very flexible & customisable lists (eg. enumerate/itemize, etc.)
\usepackage{verbatim} % adds environment for commenting out blocks of text & for better verbatim
%\usepackage{subfig} % make it possible to include more than one captioned figure/table in a single float
\usepackage{graphicx}
\graphicspath{{../images/}}
\usepackage{caption}
\usepackage{subcaption}
\usepackage{multirow} % allows multiple rows per cell
% These packages are all incorporated in the memoir class to one degree or another...

%%% HEADERS & FOOTERS
\usepackage{fancyhdr} % This should be set AFTER setting up the page geometry
\pagestyle{fancy} % options: empty , plain , fancy
\renewcommand{\headrulewidth}{0pt} % customise the layout...
\lhead{}\chead{}\rhead{}
\lfoot{}\cfoot{\thepage}\rfoot{}

%%% SECTION TITLE APPEARANCE
\usepackage{sectsty}
\allsectionsfont{\sffamily\mdseries\upshape} % (See the fntguide.pdf for font help)
% (This matches ConTeXt defaults)

%%% ToC (table of contents) APPEARANCE
\usepackage[nottoc,notlof,notlot]{tocbibind} % Put the bibliography in the ToC
\usepackage[titles,subfigure]{tocloft} % Alter the style of the Table of Contents
\renewcommand{\cftsecfont}{\rmfamily\mdseries\upshape}
\renewcommand{\cftsecpagefont}{\rmfamily\mdseries\upshape} % No bold!

%%% END Article customizations

%%% The "real" document content comes below...


\title{The Suitability of Open Street Map Data for Defining and Assessing Urban Bicycle Networks}
\author{Hugh Kelley}
%\date{} % Activate to display a given date or no date (if empty),
         % otherwise the current date is printed 

\begin{document}
\maketitle

\section{Abstract}

\tableofcontents
\listoffigures
\listoftables

\section{Introduction}

\subsection{Overview and Research Impact}

\subsection{research question}

\subsection{ethical risks}

\subsection{Research Structure}

\subsection{The role of the researcher}

\section{Literature Review}

\subsection{Cycling Behavior Research}

\subsection{Literature addressing cycling From a network perspective}

\subsection{Network analysis Literature relevant to the research methodology proposed}

\subsection{Literature Addressing the Quality of Open Street Map Data}

\section{Methods}

\subsection{Scope}

\subsection{defining networks}

\subsection{assessing networks}

\subsection{travel times}

\subsection{Accessibility}

\section{Data}

\subsection{Open Street Map}

\subsection{A Close Look at an Open Street Map Case Study}

\subsection{Transport for London Cycling Infrastructure Database}

\subsection{2011 Census Journey to work data}

\subsection{LSOA boundaries and data}

\subsection{Road KSI data}

\subsection{Data import, storage, cleaning, and joining}

\subsection{Defining Cycle Networks}

\subsection{Travel Times}

\subsection{Street Type and Street Danger}

\subsection{Accessibility}

\subsection{assumptions and concerns}

\section{Analysis \& Results}

\subsection{Defining Scope}

\subsection{Defining Networks}

\subsection{Network Characteristics}

\subsection{Travel Times}

\subsection{test for differences}

\subsection{Cycling Danger}

\subsection{Accessibility}

\section{Conclusions}

\subsection{Results}

\subsection{Limitations}

\subsection{Opportunities for improvement and extension}

\section{Appendix}

\pagebreak
\printbibliography


\end{document}
