% !TEX TS-program = pdflatex
% !TEX encoding = UTF-8 Unicode

% for a word count:
%https://app.uio.no/ifi/texcount/online.php

% This is a simple template for a LaTeX document using the "article" class.
% See "book", "report", "letter" for other types of document.

\documentclass[hidelinks,11pt]{article} % use larger type; default would be 10pt


\usepackage[utf8]{inputenc} % set input encoding (not needed with XeLaTeX)
\usepackage[backend=biber, style=authoryear]{biblatex}

% basic bib file
%\addbibresource{bibliography.bib}
% by lit type
\addbibresource{../bib/software.bib}
\addbibresource{../bib/data.bib}
\addbibresource{../Literature_Review/Literature/OSM_Quality/osm_quality.bib}
\addbibresource{../Literature_Review/Literature/Past_Similar_Work/related_work.bib}
\addbibresource{../Literature_Review/Literature/Network_Analysis/network_lit.bib}
\addbibresource{../Literature_Review/Literature/Bicycling/general_cycling.bib}


%%% Examples of Article customizations
% These packages are optional, depending whether you want the features they provide.
% See the LaTeX Companion or other references for full information.

%%% PAGE DIMENSIONS
\usepackage{geometry} % to change the page dimensions
\geometry{a4paper} % or letterpaper (US) or a5paper or....
% \geometry{margin=2in} % for example, change the margins to 2 inches all round
% \geometry{landscape} % set up the page for landscape
%   read geometry.pdf for detailed page layout information

%\usepackage{graphicx} % support the \includegraphics command and options

\usepackage[parfill]{parskip} % Activate to begin paragraphs with an empty line rather than an indent

%%% PACKAGES
\usepackage{hyperref}
\usepackage{booktabs} % for much better looking tables
\usepackage{array} % for better arrays (eg matrices) in maths
\usepackage{paralist} % very flexible & customisable lists (eg. enumerate/itemize, etc.)
\usepackage{verbatim} % adds environment for commenting out blocks of text & for better verbatim
%\usepackage{subfig} % make it possible to include more than one captioned figure/table in a single float
\usepackage{graphicx}
\graphicspath{{../images/}}
%\usepackage{caption}
%\usepackage{subcaption}
\usepackage{multirow} % allows multiple rows per cell
% These packages are all incorporated in the memoir class to one degree or another...

%%% HEADERS & FOOTERS
\usepackage{fancyhdr} % This should be set AFTER setting up the page geometry
\pagestyle{fancy} % options: empty , plain , fancy
\renewcommand{\headrulewidth}{0pt} % customise the layout...
\lhead{}\chead{}\rhead{}
\lfoot{}\cfoot{\thepage}\rfoot{}

%%% SECTION TITLE APPEARANCE
\usepackage{sectsty}
\allsectionsfont{\sffamily\mdseries\upshape} % (See the fntguide.pdf for font help)
% (This matches ConTeXt defaults)

%%% ToC (table of contents) APPEARANCE
\usepackage[nottoc,notlof,notlot]{tocbibind} % Put the bibliography in the ToC
\usepackage[titles]{tocloft} % Alter the style of the Table of Contents
\renewcommand{\cftsecfont}{\rmfamily\mdseries\upshape}
\renewcommand{\cftsecpagefont}{\rmfamily\mdseries\upshape} % No bold!

%%% END Article customizations

%%% The "real" document content comes below...


\title{The Suitability of Open Street Map Data for Defining and Assessing Urban Bicycle Networks}
\author{Hugh Kelley}
%\date{} % Activate to display a given date or no date (if empty),
         % otherwise the current date is printed 

\begin{document}
\maketitle

\pagebreak


Name: Hugh Kelley

Programme: MSc. Smart Cities and Urban Analytics

Institution: University College London

Date:

Word Count:

This dissertation is submitted in partial fulfillment of the requirements for the degree of Master of Science in Smart Cities \& Urban Analytics from University College London.

I, Hugh Kelley, confirm that the work presented in this thesis is my own. Where information has been derived from other sources, I confirm that this has been indicated in the thesis
\pagebreak

\section{Abstract}
% 158 words

Accurate representations of street networks from the point of view of an urban cyclist should account for the level of stress that the cyclist experiences when traveling on the network. When streets with a level of stress that exceed a cyclists tolerance are removed from the network, some journeys become impossible or unacceptably long. To increase cycling's share of trips in a city, building a network of infrastructure that connects destinations without exceeding a certain stress tolerance is a key priority. This dissertation implements this methodology using data from OpenStreetMap (OSM), a community generated spatial database. It finds that while difficult, it may be possible to rely on the OSM classification of streets and tagging of cycle infrastructure where professionally collected survey data is not available. Streets tagged ``Primary'' and ``Trunk'' are nearly irreplaceable in the London street network. Once these streets are removed over 40\% of journeys become impossible and the length of the remaining routes increase by approximately 30\%. 

\pagebreak

\tableofcontents
\pagebreak

\listoffigures
\pagebreak

\listoftables
\pagebreak

\section{Introduction}

	%1209 words


\begin{quote}
``When you get these incomplete networks you have this situation where great, you have this fresh new bike lane, you're excited you know?" Maerowitz said. "This infrastructure is working for you and then suddenly it's gone and you have to go back out into traffic again.'' - \cite{juhasz2019}.
\end{quote}
 
\subsection{The challenge of cycling in cities}

What makes a city comfortable for cycling? The conditions are clear when one sees them, but defining the characteristics quantitatively is challenging. 

More than half of the world's population lives in cities. Cycling is of obvious importance to the well being of the world's  urban population in terms of the macro-climate crisis, local pollution problems,  public health, wealth disparities, and urban traffic congestion. A commonly researched question is ``what factors have the most significant effect on cycling rates?'' The research has only recently begun to address cycling from a network perspective. 

A comprehensive understanding of cycling networks is hindered by multiple challenges. One is access and existence of necessary data. Another is the availability of tools for constructing data into a representational network, finally, tools for assessing and experimenting with representative networks are only beginning to emerge. 

A large body of research addresses the factors influencing the decision to cycle in an urban environment, and which factors make cycling in that environment more or less safe. Modern quantitative methods have only begun to be applied to this problem though. Challenges come through the lack of technical knowledge among interested parties, lack of access to tools for analysis of this type of complex spatial problems, and lack of access to the data necessary to define the nature of the problem and estimate the outcomes of possible solutions. 

This literature review will show that existing research is limited by two considerations. First is the scope of the analysis, considering individual changes instead of the status of the entire transportation network from the cyclists perspective.  The majority of work on this question approaches the matter from a perspective of discrete policy intervention and the effect of marginal improvements on cycling. This fails to build a comprehensive theory of cycling rates that a network analysis approach can offer. Only network theory can offer a high level look at the level of service in a community. Second is the difficulty of obtaining data used by the studies that do take a comprehensive network approach. The more recent and more network orient studies used data collected and controlled by local governments. For communities to advocate organically for improvements to cycling networks, there needs to be a data source and methodology that is free and open to all. 

As seen in the analysis section, Open Street Map has the potential to offer this but that data the presents a number of challenges to someone using the map for the analysis of cycling networks. 

 The implication of this view is that infrastructure changes are unlikely to have a meaningful effect on behavior in the system until a critical mass or tipping point is reached in the network. 

Basic attitude is that improvements to infrastructure only matter to the extent that they improve safe connectivity to ``important'' nodes/edges. 

In this context, this work hopes to make a contribution using London as a case study. London is an attractive case because it has a considerable cycling population but does not yet have the level of infrastructure that exists in world leading cities like Copenhagen. Thus London's position at a mid point of cycling infrastructure development allows strengths and weaknesses to a live program of improvement to be identified. 


%%%%%%%%%%%%%%%%%%%%%%%%%%%%%%%%%%%%%%%%%%%%%%%%%%%%%%%%%%%%%%%%%%%%%%%%%%%%%%%%%%%%%%%%%%%%%%%%
\subsection{Research Question}

\subsection{Key Questions}
what was the scope and how was it defined
what filters were used
what were the walking and cycling speeds used
what were the travel times calculated
what was the directness of travel on different networks
what is the distribution of travel times in the different networks
what lsoas changed the most and least across networks
how did removing road types affect the number of lsoa centers connected?
how long did the calculations take?
Did removing directionality from the networks have a significant effect?
Did removing directionality mediate the effect of removing primary and trunk streets?



What is the quality of the London Cycling network? 

How accurately does Open Street Map represent London's streets.

What are the challenges of using Open Street Map to create networks representing the cycling network from the persepctive of different types of cyclists. 

How resilient is the London street network to the removal of it's most important, busiest and dangerous streets? 
	Connectivity
	Directness
	
How does travel by bicycle compare to travel by public transport in London. 




This research intends to define the overall suitability of a city to cycling using estimates of the roads cyclist are and are not willing to use and the effect these decision have on travel times. The central research question is: \textit{how does a preference for safety decrease accessibility in central London when traveling by bicycle?} It extends the literature in three ways; extending quantitative research to a holistic approach rather than focusing on individual edges or nodes, extending holistic approaches to quantitative outcomes rather than visual or anecdotal conclusions, and by emphasizing the use of open source tools and data that are available to any member of community. 

The research provides a methodology that allows a community to identify nodes and edges for improvement that will have the largest impact on the overall network, quantify the expected improvement, and thus require from elected officials specific changes to the roads in their communities. 

The subquestions addressed in the methodology are, 

What is the quality of the Open Street Map metadata for streets in London? 

How important are the OSM highway types to efficient transit within London? 

How does transit by bicycle compare to transit by public transport within London? 

The issue implied by these research questions is: to what extent is street level transportation space a zero sum game? 

Can a high level of service for cyclists be built without taking large amounts of important space away from automobile trafic? 

The intended outcomes of this research include(1) an understanding of the roles of location and connectivity for cycling infrastructure and (2) an understanding of the usefulness of Open Street Map data for estimating these sorts of measures. 


\subsection{ethical risks}

This project relies entirely on publicly accessible data. For this reason, ethical risks are not present in the research methodology.  

\subsection{Research Structure}

First, existing work on this research area will be reviewed and the techniques to be built upon will be identified.  Then a methodology for defining the strength of cycling infrastructure in a city will be specified. The data available for analysis will be described in the context of past work, and what is available for other cities and through other channels that were not available to this research. the steps taken for data cleaning, transformation and joining will be specified. Section 3 will describe the implementation of this methodology for London, identifying the scope of the case study, defining the exact data collected and transformed and the specific tools used. 

The multiple stages of results will be reported, and interpreted. Finally conclusions will be drawn, areas of further research specified, opportunities to improve the methodology and the quality of the data emphasized and the key recommendations for further improving the London cycling infrastructure network will be made. 

After critically reviewing this existing research, a methodology for using open source tools and data to estimate the relative strength of a cycle network and a method for prioritizing improvements will be specified.

The review that follows takes the structure of an initial look at typical literature considering how to promote cycling, recent attempts to use network analysis to accomplish this same goal, and closes with a look at work in network analysis that can inform applying the field to cycle networks.

\subsection{The role of the researcher}

Read other versions of this and see if it makes sense for this 




\section{Literature}

	
\subsection{Cycling Behavior}

The most important conclusions from literature that tries to predict cycling behavior is the focus on multiple types of cyclists, and the factors that influence each type's decision to cycle. Most often, four types of cyclists are used, ``strong and fearless'', ``enthused and confident'', ``interested but concerned'', and ``no way no how'' (\cite{dill2013four}). This categorization is sometimes changed to follow demographics, focusing on young, generally male, adults, older adults, and childern. CITE. The literature separates the decision to cycle into the decision of cycling frequency, how often someone who is willing to cycle generally chooses to do so, and the decision to cycle at all, with different factors influencing each decision (\cite{stinson2005comparison}). 

Despite behavioral differences between these categories, research has shown that all cyclists are willing to sacrifice time and energy for increased safety on their route. CITE. Indeed, psychological research showed that fear is a significant factor during an urban cycling trip (\cite{ellett2018state}). This is important given the sensitivity cyclists show to efficiency. CITE. The impact of a route change can be very significant, for instance a higher frequency of stop signs on a route can double the energy required for a journey (\cite{fajans2001bicyclists}). Given the common trade off between efficiency and safety, it was found that the effect of infrastructure improvements is very dependent on context, effect is a function of the change in safety, and the importance of the location to trips (\cite{kondo2018bike}), improvements that meaningfully increase safety at important locations have the largest effect. 

Several studies looked at the importance of perception in behavior change, assuming that a real change in safety is irrelevant if it is not perceived by potential cyclists as a change(\cite{li2012physical} and \cite{parkin2007models}). This gives rise to literature that focuses on a behavior and attitude change approach from psychology the prioritizes change in habits and perception over infrastructure, with changes to the built environment only used where required to change perception (\cite{savan2017integrated}). This should be considered in the context of research showing that cyclist perceptions of danger are generally accurate (\cite{vandenbulcke2014predicting}). Thus it seems reasonable to conclude that despite the decision to cycle being a fairly complex mix of factors, the decision for almost any urban commuter, comes down to safety, and efficiency. An efficient network of safe edges, connecting important nodes, would be expected to have a meaningful effect on the rate of cycling in an urban area.  


%%%%%%%%%%%%%%%%%%%%%%%%%%%%%%%%%%%%%%%%%%%%%%%%%%%%%%%%%%%%%%%%%%%%%%%%%%%%
%%%%%%%%%%%%%%%%%%%%%%%%%%%%%%%%%%%%%%%%%%%%%%%%%%%%%%%%%%%%%%%%%%%%%%%%%%%%
\subsection{Cycling Networks}

\cite{buehler2016bikeway} is a very useful introduction to the literature on cycling networks. Unfortunately it concludes that very little true network analysis has been developed for cycle networks. The majority of papers they found could be categorized into those that focus exclusively on nodes, and those that focus on edges of the dual graph, where intersections are nodes and streets are edges. At the time of writing there were 115 papers citing Buehler's review, however all but 5 of them fail to take Buehler's central recommendation that \textit{If individual characteristics of a network's links and nodes contribute to cycling levels, it logically follows that a network of such features would as well...}. The ``Toward Studying the Whole Bicycling Network'' section of Buehler's review is a good overview of attempts up to 2016. The key findings were that continuity and connectivity of infrastructure is valued by cyclists. Of particular interest is the \cite{schoner2014missing} study of the relationship between network characteristics and cycling mode share in 74 US cities. That study found density of the network had the highest elasticity for effect on cycling rate. 

Several of the works reviewed contribute new ways of measuring ``quality'' of the infrastructure, these quality measures include a Bicycle Compatibility Index (BCI) (\cite{klobucar2007network}),  Bicycle Level of Service (BLOS) (\cite{lowry2012assessment}), and Level of Traffic Stress (LTS) (\cite{mekuria2012low}). Each of these can reasonably be viewed as an attempt to measure the ``safety'' of a network link. These studies generally lacked a rigorous method for prioritizing nodes by importance or defining a sample set of trips between nodes. Improvements to this will be addressed in the section reviewing network analysis literature. 

Buehler notes that a key challenge to using the network methods reviewed is data availability, this research hopes that network analysis can be a method for reducing rather than extending the amount of data necessary to understand a cycle network, as network statistics could be used to replace some empirical measurements as discussed below.  They further criticize the approaches as lacking empirical validation. Gathering accurate cycle traffic data and safety data is an immense challenge as demonstrated by the flow estimation techniques of \cite{gosse2014estimating} and the safety estimate techniques of \cite{puchades2018role}, which focuses on near misses as a proxy for predicting actual safety incidents, but acknowledges the difficulty of collecting near miss data without human observation. CITE. 

Since the publication of Buehler's review, the papers extending the full network analysis method have had moderate success. \cite{akbarzadeh2018designing} uses taxi trip data to weight the links between destinations in order to build communities of nodes that tend to be origin and destination pairs. While this is a novel approach to prioritizing edges, taxi usage tends to be for less frequent travel between nodes and the general literature as well as this work focuses on daily commuting, which is rarely accomplished by taxi. \cite{boisjoly2019bicycle} focuses on the directness of routes on cycle paths between nodes. 

\cite{doorley2019designing} focus on building cycle infrastructure to maximize a function of travel costs, infrastructure costs, health, traffic accidents, and pollution. While this is an interesting approach, it addresses a more political question in the sense that the key result of the algorithm is to recommend a specific amount of investment in cycle infrastructure to maximize the costs and benefits to all road users, The author's fail to recognize the prioritizing the goals of public policy is a highly normative and subjective exercise and that a model designed to give an ``objective'' answer to this question inherently reflects the author's preferences and when calibrated to ``the real world'' reflects the biases and preferences of the status quo, rather than the true ideal outcomes preferred by a population. Instead, modeling, especially for urban planning purposes, should accept an exogenous goal and implement it as efficiently as possible. For instance, cycle infrastructure, is explicitly intended to reduce motor vehicle use, it would make no sense to then use a model that determines the efficient level of motor vehicle use, the political process has already determined the answer and merely asks for implementation recommendations from the modeler. 

\cite{mauttone2017bicycle} similarly focuses on an optimization framework for cycling networks, choosing a subset of streets that are ``suitable to building cycle infrastructure''. This is odd in the sense that the goal of building cycle infrastructure is to \textit{create} streets that are suitable for cycling, not merely identify them. Similar to \cite{doorley2019designing}, they identify a cost to building cycle paths which they seek to balance against the benefits. 

Overall, it is not clear that a model for building cycle paths should be particularly cost sensitive. \cite{gu2017cost} found a very high return on investment to the budget for cycling infrastructure in New York City. The very idea of using network analysis for the development of cycle networks emphasizes the potential non linearity of the effect of building more infrastructure, with usage accelerating as the network approaches ``completeness'' in some for. In addition, cities tend to combine cycle infrastructure improvement with other required improvement and maintenance activities, mitigating the costs by being opportunistic in implementation. 

Lastly, \cite{osama2017models} uses a number of predictors including network statistics to predict bike travel within zones of Vancouver similar to \cite{schoner2014missing}. They found a positive coefficient for the density of the bike network in the zone. 

Thus while network analysis has been applied to cycle infrastructure, clearly there is not a consensus on the methodology. In particular, a definition of ``quality'' or ``safety'' has not been established. Additionally, a method for exploring the network of infrastructure defined is still lacking. Lastly,  


\subsection{Network analysis Literature relevant to the research methodology proposed}

\textit{note: this section should probably be worked into the review of cycle network analysis above instead of separated out. Combine criticisms of existing work and solutions from network science. Or it might just have to come out.}

This work intends to improve upon the work detailed above through the inclusion of more ideas from network science that can improve and extend and simplify the models already specified. 

There are a number of considerations from network analysis literature that can complement the work already applying the field to cycling networks. Perhaps the most important is the idea that the transportation network and the location of activities in a city reinforce each other. The most productive activities that occur in a city tend to be the most easily accessed according to central place theory. This has been shown empirically for instance by \cite{porta2012street} and by \cite{wang2011street}. In the literature discussed above authors address the key change in networks of defining a computationally feasible set of trips between nodes in a few different ways. This is misguided though as network theory offers a number of tools for exploring a graph efficiently. A random walk weighted by a centrality measure will tend toward trips between more important nodes because these tend to be more central. For instance \cite{Jiang2009characterizing} or  \cite{volchenkov2007random} both show that weighted random walks can do a accurate job of representing human activity on a street network. 

A number of network statistics are used in the literature reviewed above. \cite{crucitti2006centrality} complements these as the centrality and connectedness measures used should be more formally justified according to network theory. 

While, the multiplex offers an interesting approach to studying the interaction between bicycle usage and train and bus, this is beyond the scope of this work and perhaps beyond the scope of London , where bicycles are not generally allowed on public transport, unlike some other cities. 

The potential for a ``transition'' in the cycle network per \cite{barthelemy2018transitions} is an exciting possibility. Finding edges that need to be added to a network in order for a transition in network connectivity to occur is a valuable exercise. In addition, specifying network characteristics according to those specified by \cite{barthelemy2011spatial} is also very valuable. 

Overall there are a number of considerations from network science that this research intends to add to the analysis of cycle networks. The exact implementation of these considerations will be specified in the methodology section.


\subsection{Open Street Map Data Quality}

Citations

test citation for \cite{mobasheri2017crowdsourced}, did it work?

considerations
much harder to find errors of omission than inaccuracies and typos

most studies use anothe rhigh quality data set like the ordnance survey to judge quality

most studies also look at the accuracy of feature locations, rather than the accuracy of the metadata

meta data tends to be analyzed for data source and history, rather than to look at the characteristics of a given feature. 

One particularly useful study looked at OSM data in the context of routing for the disabled. "Are Crowdsourced Datasets Suitable for Specialized Routing Services? Case Study of OpenStreetMap for Routing of People with Limited Mobility "

the conclusion was: "Based on empirical results, it is concluded that OSM data of relatively large spatial extents inside all studied cities could be an acceptable region of interest to test and evaluate wheelchair routing and navigation services, as long as other data quality parameters such as positional accuracy and logical consistency are checked and proved to be acceptable."  

which is a good indication for this work 

can also cite Zielstra 2013 on the effect of large data imports on OSM comleteness in the context of the TfL CID thing that was just released. 


A large body of work has considered the challenge of quantifying the quality of Open Street Map Data. A large part of this work focuses on the spatial accuracy of the data, how closely are features mapped to their real locations. As the quality and coverage of OSM expands, the challenge in data quality shifts from quality in the spatial part of the data set to quality in the metadata, that indicates the source of the feature and characteristics of that feature. 

\section{Methods}

	% 1254 words


%%%%%%%%%%%%%%%%%%%%%%%
%- define scope, minimize area while capturing the largest number of trips by bicycle, trips in total, road casualties, and households and jobs. This will be some part of central London. 
%-- build boundary based on scope using postgis and qgis
%- define filters for OSM data
%-- this includes trying to understand the relationship between relations and individually tagged edges and nodes. 
%-- pictures of network, specific edges, and photos of real life infrastructure. 
%-build networks from data
%-- removing streets
%-- undirected
%- identify network nodes closest to lsoa centroids
%-compare \% of trips possible across Quant, bike 1, bike 2, and undirected versions. 
%- compare change in directness between bike 1 and bike 2.  
%-- compare bicycle accessibility to Quant
%-UTM-30 projection used as standard across all data. 
%%%%%%%%%%%%%%%%%%%5

\subsection{Scope}

The scope of the study will be defined so as to maximize the study's representation of London's cycling commutes, capturing as many trips by origin and destination as possible as well as considering the location of road casualties in London. This will be done in the context of the computing resources available, where travel time calculations for the origin destination matrix must be reasonable. 

\subsection{Defining Networks}

The first step in this investigation is to build a data set that represents the London cycling network as accurately as possible. This representation needs to reflect the fact that different cyclists are willing to use different streets as a function of the perceived safety of the street and the level of confidence of that cyclist. Thus the data set will be multiple representations of the city cycling network that each represent a level of confidence, only including streets with a certain level of safety. 


A key question then is, how to quantify ``safety''. In a perfect world, this would be done empirically. This would involve a combination of cycle traffic volume collection, cycle traffic behavior observation, and interviews with a representative set of cyclists and non-cyclists about their decision making. All of this data could be compared to the cycling environment in different locations to find cyclist sensitivity to different factors. 

Networks will be defined as a subset of edges in the Open Street Map network. 

Download the connectivity data from OSM via the Overpass APIas json . A graph is built with the nodes and ways from the Overpass data with edges coming from the Overpass data's ``way'' elements, and the nodes coming from the intersection of the edges as defined in OSM as well as the end of an edge. Getting to the point requires simplification because the Open Street Map data includes many more nodes than just the endpoints and intersections. The network is simplified by removing nodes with degree 2, where the node simply connects one street to one other, unless there is a difference in directionality between them. That is, when a street changes from 2 way to one way, a node is placed at the intersection to denote that change. 

Cite OSMnx \cite{osmnx} and Networkx \cite{networkx}

How were the representative networks built? 

Open Street Map uses tags to associate street characteristics with the geometries that make up the map. Appendix XXXX contains the definitions from the OSM wiki page for each of the tags used. Most important to note is that this research uses 5 ``levels'' of street stress. The highest level allows all street conditions. The second level allows all but ``primary'' streets, the largest busiest streets. Then ``secondary'' streets are removed in the third filter. ``Tertiary'' type streets are moved to build the fourth filter. The fifth filter moves ``living streets'' and ``residential'' streets which both are specified to be low traffic streets used primarily for local trips. Thus the most conservative layer of the network contains only edges where there is no expectation of interaction with motor vehicles. 

\subsection{Defining Origins and Destinations}

QUANT uses the node of highest degree in a given LSOA. 

For calculating a sample of routes on the networks, this analysis used the node closest to the centroid of each lsoa. 

\subsection{Assessing Networks}

Test citation for \cite{osmnx}. Did it work
Test citation for \cite{networkx}. Did it work
Test citation for \cite{qgis}. Did it work
Test citation for \cite{python}. Did it work
\subsection{travel times}

Travel times will be calculated for each network with a significant portion of the area of investigation connected by a single component.

In addition to calculating the multi-directed graphs for different OSM filters, travel times are calculated for undirected versions of the graph. This will be used to investigate how building infrastructure for cyclists to travel safely against traffic could further raise accessibility or replace the need to build infrastructure that allows a cyclist to safely travel on main roads. 

 This is for comparison between filters, how do travel times change with the inclusion or exclusion of street types. This is also for comparison with public transport times within the area of investigation. 

A key point of interest is the networks that result from different filters. For different filters, the count of nodes and edges, and therefore the overall density of the network will be considered. The number of independent components will be considered. 

While this work is unable to construct a continuous measure of traffic stress for each edge, the OSM tags for a given edge will be used to conduct a percolation type construction of the graph, adding edge types. The change in largest connected component will demonstrate the importance of the different types of edges to the network. 

Compare the mean of the distributions and compare the shape of the cycling distribution to the shape of the QUANT transport distribution. 


\subsection{Accessibility}

a connectivity ratio will be calculated following \cite{furth}. 

Visualizations of travel times will be built to understand whether there are particular areas within the scope of investigation with better or worse efficiency. 



\section{Data}

	% 1335 words

\subsection{Open Street Map}

%Intro and overview

Open Street Map is a mapping project started in 2004 to collect volunteered geographic information. It consists of geometries drawn by users, either in person, as they travel through a city or remotely, looking at donated satelite images of cities. Open street map provides community generated geospatial data. This data is accessible via the overpass API from several hosts. Geoff Boeing describes using the OSM API query language as "notoriously difficult" (\cite{osmnx}). 


%Data Structure


Second to the actual geometry of a "way", streets and paths, node, single point on the map, or relation, collection of ways nodes and other relations are tags. Tags specify what a particular geometry is, what its characteristics are, and rules for use or other characteristics of the geometry. This allows for differentiation on the map between public and private areas, specification of what exactly a node is referencing, an intersection, mailbox, or a business location, or the type of traffic allowed or commonly seen on a street way. 


% Possible Problems with Data

As discussed in Literature reivew section XXXXXXX
 
 
 There are four possible problems with Open Street Map, missing geometries, inaccurate geometries, missing tags and inaccurate tags. The review of literature will provide detail on attempts to measure the accuracy of the data in Open Street Map.
 
 
% General Way to build data by filtering. 



Compare Relation = cycleway to a list of edges and nodes tagged cycleway

Adding living streets and residential streets don't do much. 

Part of the problem is the lack of consistent tagging, it only takes one line segment missing a tag to disconnect two nodes,

but this also reflects the fact that getting somewhere within London nearly always requires leaving cycle infrastructure at some point and using main roads. 

Two basic problems for using OSM data to define cycling networks were found. The first is that defining the network by relation exagerates the network, the second is that relying on the metadata tags of individual features understates the cycle network. Looking at OSM it is clear that the relations identifying bike routes are not reflecting sets of ways and nodes with a given tag, or reflecting streets and intersections where cycling is meaningfully safer than other streets and intersection. At the same time, it is clear that there are many ways and nodes that are more accomodating to cyclists than their OSM metadata indicates. 


Cite OSM wiki page
https://en.wikipedia.org/wiki/OpenStreetMap


%%%%%%%%
% image of OSM London cycle network by relation
% image of OSM London cycle network by tag
% image of OSM missing tag situation from Overpass turbo
% image of same location street view
% image of OSM implied cycle infrastructure
% image of real street without cycle infrastructure. 
%%%%%%%%%%%%%



In the case of the relation, Quietway XXX was examined in person. figure XXX shows an image of Brandon rd, a part of the quiet way. this way is tagged
Brandon rd is part of OSM relation XXX the Hackney Camden cycle route.  this road though, as can been seen in the image, has no actual cycle inrastructure. This in Open street map, it is tagged as an unclassified highway. There is a tag noting the max speed is 20 mph. Data collected for 20 mph streets found that as many as 80\% of drivers exceeded these limits. 
https://www.thesun.co.uk/news/7253694/20-mph-zones-cause-more-deaths/

In other cases, osm underestimates the quality of cycling infrastructure. for instance the intersection of Mile end Road and Cambridge Health Road in the borough of tower hamlets is a high traffic intersection both for automobiles and for cyclists. It is an integral part of the Stratford to Aldgate cycle super highway. This intersection has been reworked to be safer for cyclists. In open street map though, it i labeled a "trunk" highway, due to its high traffic nature. It is way 7058092014. There is also a tag cycleway:left=lane indicating that there is a cyclelane on the left side of the street. 

\cite{osm}

\subsection{A Close Look at an Open Street Map Case Study}

\begin{figure}
\centering
\includegraphics[width=0.6\textwidth]{brandon_rd_cropped}
\caption{image of quietway}
\end{figure}

\begin{figure}
\centering
\includegraphics[width=0.6\textwidth]{euston_rd_cropped}
\caption{An image of Euston Road}
\label{fig:euston}
\end{figure}

\subsection{Transport for London Cycling Infrastructure Database}

\cite{tflcid}

This data was in the process of being publishing during the period of research for this work. While it was not available at the time of publication, it is notable because it promises to raise the quality and volume of OSM data for London substantially. It includes 2,000km of cycle lanes as well as hundreds of thousands of parking spaces, cycling related signs and other relevant features. \cite{osmtflcidwiki}. 

Section XXXX from the Lit Review addressed to possibility for bulk data uploads to dramatically enhance OSM data quality and this may be one example. The data was professionally surveyed. 

\subsection{QUANT}

The QUANT dataset of public transport travel times was .

CITE

The of 22 million pairs, 5.8 million pairs had no public transport link. To clean this data, the set is cut down to match the scope of the the investigation. Where there is no link between an origin destination pair, a link is constructed by combining the walking time from the origin to the node of highest degree in another LSOA where public transport is available to the destination LSOA. 

The walking speed used will be taken from google and the distance is the straight-line distance between two points. This is less accurate than actually finding the walking route between the two points but this level of detail was not computationally feasible. 

The QUANT data provides a point of comparison for cycling travel times to be calculated. 

\subsection{2011 Census Journey to work data}

The 2011 census asked each household where they lived, where they worked and how they traveled to work the preceding week (\cite{jtw}). Thus data for origin and destination by mode of transport was available. This data will be used to help define the optimal scope as well as to compute a ``connectivity ratio'' like that of \cite{furth2012low} for a network. 

This data is shared through the Nomis Labor Force website as multi-sheet excel pivot tables. Making the data usable required, stripping the meta data headings from each sheet, importing the book to pandas dataframe by sheet, melting from a pivot table to long data with origin, destination, and count columns, adding the sheet name that identified the mode of travel as a column, appending each individual sheet together into a single dataframe, and pushing the dataframe to the sql database. 

\subsection{LSOA boundaries and data}
	
LSOA boundaries were obtained from  \cite{lsoageoms}. The LSOA boundareis were determined as a part of the 2011 census containing approximately 5000 people each. These are used, first to specify a boundary for the scope of the study and then to specify origin and destination nodes on the network. The node closest to the centroid of each LSOA is selected as the origin and destination or that LSOA. Where LSOA's were comprised of multiple polygons, the centroid of the largest polygon was used. These polygons were used for the production of maps seen in the Analysis section XXXXX. 

On maps created using these boundaries the copyright must be stated. This is
%"Contains National Statistics data © Crown copyright and database right [2015]" and "Contains Ordnance Survey data © Crown copyright and database right [2015]"
	
\subsection{Road KSI data}

Data on those killed or seriously injured on London streets is available through CITE. This was used to assist in determining the scope of the investigation. While it was hoped that the data could be used to build an estimate of danger to cyclists on London's streets several obstacles prevented this. The first is the change in infrastructure over time and lack of data about the exact infrastructure present at the time and location of each incident. Second was the lack of high resolution data about cyclist volumes. An area that has a particularly high KSI rate may be especially dangerous or may be relatively safe after adjusted for cyclist miles traveled, an unknown. London has begun collecting some data on cyclist volumes although this remains fairly sparse. \cite{cyclistksi}
	
\subsection{Data import, storage, cleaning, and joining}

data import was done in python using the csv, pandas, geopandas, json, and osmnx packages. 
Data from OSM was converted from json to a dataframe, tags simplified, edges truncated, geometry simplified. 
Geometry was converted to well known text and multi-lines were broken into several single line geometries for compatibility with the postgis databse. 

Once the data was cleaned, it was passed to a postgres database with the postgis extension using the sqlalchemy package. Postgis was used for calculating distances, associating nodes with centroids. 

remove multipolygon lsoa interior rings in favor of polygon lsoa shapes

QGIS was used to melt polygons into outer boundary and for the construction of visualizations

cite postgres
\cite{postgres}
cite postgis
\cite{postgis}
cited dbeaver
\cite{dbeaver}
cite python
\cite{python}
cite pandas
\cite{pandas}
cite json

cite sqlachemy
\cite{bayer2010sqlalchemy}
cite qgis
\cite{qgis}

\section{Analysis \& Results}

	% 1907 words

\subsection{Key Questions}
what was the scope and how was it defined
what filters were used
what were the walking and cycling speeds used
what were the travel times calculated
what was the directness of travel on different networks
what is the distribution of travel times in the different networks
what lsoas changed the most and least across networks
how did removing road types affect the number of lsoa centers connected?
how long did the calculations take?
Did removing directionality from the networks have a significant effect?
Did removing directionality mediate the effect of removing primary and trunk streets?



\subsection{Defining Scope}


Goal is to capture the largest computationally feasible network with a simple set of rules. 

The first rule was to restrict the network to ``inner london''. This has the advantage of a formal designation by the GLA for each borough, capturing, XXX\% of the population with a population density of XXXXXX compared to XXXXX for london overall, XXXXXX\% of the jobs in the city, and XXXX\% of the journey's to work. Additionally, rates of cycling are higher in inner London than in the periphery. 

Second, the area of interest was restricted to north of the river Thames. This captures XXX\% of the population, with a density of XXXXXXX, XXX\% of London's jobs, and XXX\% of the journey's to work. Further, it has the advantage of excluding the need to cross the river, where journey's are focused on a few number of bridges, with a significant effect on the shortest paths, reducing the effect of other changes on the network. 

\begin{table}[]
\centering
\begin{tabular}{lcccl}
 Mode Share Within Scope & All Modes & Bicycle & \% by bicycle &  \\
 \hline
 Origin in scope &  981,354 & 46,832 & 4.8\% &  \\
 Destination in scope & 1,454,606 & 48,461 & 3.3\% &  \\
 Both in scope & 479,882 & 24,843 & 5.2\% & \\
 All journeys & 5,852,298 & 140,180 & 2.4\% \\ 
\end{tabular}
\caption{commuter data}
\label{table:commute_data}
\end{table}

text


\subsection{Defining Networks}

There are three possible ways to construct a set of ways and nodes from OSM. These are; a positive filter; a negative filter; and selecting by relation. A positive filter specifies tags that an way or node must have to be included. A negative filter includes all ways and nodes without the tags specified. A relation is the OSM term for a collection of ways and nodes that belong to a set identified by an OSM contributor. 

OSM relations in London identified as being cycle routes are mapped in Figure XXXXX. The query for selecting this set can be found in the appendix XXXXX. It is clear that this is an extensive set of geometries. Clearly though the density of that network is fairly low and for almost any trip a user would need to venture beyond the relation geometries. Further, many of the ways included in the cycling relations are in fact no different from other streets. As seen in figure XXX a picture of Brandon Way. This has no safety improvements for cyclists, it has been observed that speeds can far exceed the 20 mph limit. Thus, the cycle relations on their own are insufficient for building representative networks. 

A positive filter was explored as a way to include exactly the edges and nodes that met a given level of stress for cycling. This method was challenged by the fact that a single road segment not tagged accurately would disconnect a network part. Further, there was not a method found to work correctly for selecting roads with any one of many tag values.  Table XXXX details all of the possible tags related to cycling that were found in the London OSM data.  In the Overpass Turbo application, this could be accomplished by using multiple subqueries as detailed in the Overpass Turbo Cycling network query listed in Appendix XXXX. In OSMnx this multiple queiry statement structure was not an option. Thus a method for building full and accurate networks of OSM geometries using a positive filter was not found. 

The negative filter, implemented as a modification of the filters built into OSMnx version 0.11dev was the most successful. The first filter included all ways and nodes not explicitly tagged with values indicating that cycling was not allowed. Row XXXX of Appendix XXXX. The second filter was the same as the first but excluded streets tagged as primary and trunk. These are the two tags used for the highest priority street types as seen in Table XXXX describing standard tags for the ``highway'' key in OSM. The third filter restricted secondary streets in addition to the restrictions of the second. The fourth filter restricted tertiary streets leaving only living and resdiential streets in a addition to non-motor vehicle ways like canal paths and segregated cycle lanes. The final filter, 5, restricted all edges where a cyclist might interact with motor vehicles. 

Lastly, two more networks were specified. These were networks 1 and 2, all streets and all streets but primary and trunk, with the directionality of the streets removed. These networks would be used to test the effect of direction restrictions on travel times and whether removing direction restrictions for cyclists could successfully replace the need to use some more busy and dangerous streets. 

The weakness of the negative filter was the inability to include streets tagged as multiple types. Negative filter 2 excluded any highway tagged primary regardless of whether it was also tagged with cycleway or cycleway:left=lane. Thus the negative filters also do not fully reflect the reality of the London street network for a cyclist. However, it was the best method for testing the importance of street types and the effect of directionality on travel times. 


Each network is created using a filter that excludes Open Street Map features tagged with certain values. All features tagged with ``bicycle=no'' or ``service=private'' were excluded. Additionally, where the edge's ``highway'' tag was ``footway, steps, corridor, elevator, escalator, motor, proposed, construction, abandoned, platform, or raceway the feature was also excluded. 

good example of the difficulty of building a good network representation is castle baynard st. It connects the central london part of cycle superhighway 3 with the east london section that continues out to limehouse. this is a tunnel that serves as a bike path and as a driveway of sorts to an underground car park. 

insert picture of castle baynard street. A network built from the relation[route=bicycle] set of ways and nodes would include this but the negative filters do not. 


"to measure miles of designated bike facilities can be misleading. Some designated bicycling facilities involve LTS values that most people will not tolerate. "\cite{furth}

This is the challenge to using the London designated cycle routes. the routes do not necessarily reflect safe cycling.

\begin{figure}
  \centering
  \includegraphics[width=0.5\linewidth]{bbox_bike_1_filter_cropped}
  \caption{1: most confident cyclists}
  \label{fig:sub1}
\end{figure}

\begin{figure}
  \centering
  \includegraphics[width=0.5\linewidth]{bbox_bike_5_filter_cropped}
  \caption{5: no interaction with cars }
  \label{fig:sub2}
\end{figure}



\begin{figure}
  \centering
  \includegraphics[width=0.5\linewidth]{bbox_bike_2_filter_cropped}
  \caption{2: all but primary and trunk streets}
  \label{fig:sub2}
\end{figure}



\begin{figure}
  \centering
  \includegraphics[width=0.5\linewidth]{bbox_bike_4_filter_cropped}
  \caption{4: only residential and living streets}
  \label{fig:sub2}
\end{figure}
 

\subsection{Origins and Destinations}

One origin/destination node was selected for each LSOA. The node was selected as the node from the set of nodes across all network definitions that was closest to the centroid of the LSOA polygon. This is essentially a sampling technique. While individual nodes may give strange results due to the specifics of their locations, it is expected that the average results for 894 nodes that yield 798,342 origin destination pairs will be a sufficiently large sample that individual idiosyncracies balance out. 

As seen in figure XXXX the distribution of the differences between the distance between centroids and the distance between actual nodes is well balanced. 

\begin{figure}
\centering
\includegraphics[width=0.5\linewidth]{node_centroid_dist_diff}
\caption{distribution of differences between distances between nodes and distances between centroids}
\label{table:dist_diffs}
\end{figure}




\subsubsection{Quant Network}

Quant has 23,377,225 pairs. The subset has 799,236. The subset has 200,041 missing distances. 

To complete the missing distances, the quant travel times were built into a networkx multidigraph. There was an edge for each node pair that represented the walking time between the nodes, calculated as the straightline distance between the nodes divided by the google walking speed 3 mph converted to 4.83 kph and converted to  0.0805 kilometers per minute giving a number of minutes walking time between the nodes that was in the same units as the number of minutes public transit time between the nodes. 

Then for each pair of nodes missing a transit time, the shortest path on the graph was calculated. Thus each travel time can be a combination of walking and riding public transit between nodes. 


\subsection{Network Characteristics}

chart of largest connected component of network as edge types are included. 

\begin{figure}
\centering
\caption{largest component of network type}
\label{fig:connected_component}
\end{figure}

no cars,
+ living streets
+ residential streets
+ tertiary 
+ secondary
+primary

\begin{table}
\centering
\caption{table of network statistics}
\label{table:network_stats}
\end{table}

\subsection{Defining Origins and Destinations}

Node 5816785884, closest to the centroid of LSOA XXXXXX is found at the entrance to a garage at the end of a one way street. Thus every other node in the network is inaccessible on the directed versions of the network. The edge leading to this node is tagged ``service'' so perhaps service streets should have been excluded as they are frequently dead ends. 

\begin{figure}
\centering
\caption{histogram of distances between centroid and nearest common node. }
\label{fig:centroid_node_dist_hist}
\end{figure}

Because urban density increases as one approached the center of a city, there may be a slight bias towards node distances being lower than centroid distances. This is because there is a higher probability that the closest node will be on the central side of the centroid than the outside. 

\begin{figure}
\centering
\caption{distribution of differences between Quant distances and Cycle Origins and Destinations}
\label{fig:diff_dist}
\end{figure}

\subsection{Travel Times}

Travel times were converted from distances using a walking speed of 3 mph and a cycling speed of 8 mph based on data taken from google maps estimates for journeys as seen in that table. These were converted to kilometers per minutes, waking: $0.0805$ and biking $0.215$. 

\begin{table}
\centering
\caption{Google Maps travel speeds}
\label{table:travel_speeds}
\end{table}

As seen table XXXXX travel times for bike network 2, without trunk or primary highways, are substantially longer than travel times for bike network 1. This indicates that the effect of removing these edges is not just to disconnect the network but also requires a network user to take a less direct route, straying further from the straight line between origin and destination. 

The difference between directed and undirected distances is also notable. 

the standard deviation 

the min is unchanged, while the max 

Travel times by bicycle for aggressive cyclists are XXXXX compared to the QUANT travel times by public transport. 


Transit times are consistently lower than cycling times. However, transit times are from hub to hub, so there is in most cases, an additional walking time to get from the transit hub to the actual destination that would be less of a factor for a journey by bicycle. 


The connectedness estimated may understate the true connectedness. This is seen in that unconnected node pairs in almost every case have one node that is unconnected from all other nodes. Thus there are not local islands of connectivity, there are nodes being disconnected from the network due to the removal of a single edge. 


\subsubsection{test for differences}


\begin{table}
\centering
\caption{changes between networks, \% connected and $\Delta$ directedness}
\label{table:change between nets}
\end{table}

There's got to be a significant difference between the distances for the different networks right?


\begin{table}[]
%\centering
\begin{tabular}{lrrrrrrrrr}
network                      & u\_1 & d\_1 & u\_2  & d\_2  & quant & quant+ & d\_3 & d\_4 & d\_5 \\ \hline
                             &      &      &       &       &       &        &      &      &      \\
largest component edges      &      &      &       &       &       & -      &      &      &      \\
connected pairs              &      &      &       &       &       &        &      &      &      \\
average directness *         &      &      &       &       & -     & -      & -    & -    & -    \\
min travel time*      & 0.4  & 0.4  & 0.4   & 0.4   & 0.1   &        & -    & -    & -    \\
mean travel time*     & 33.0 & 33.8 & 41.1  & 47.4  & 22.6  &        & -    & -    & -    \\
max travel time*      & 88.8 & 91.2 & 114.0 & 124.0 & 51.9  &        & -    & -    & -    \\
travel time std. dev* & 16.8 & 17.1 & 20.4  & 23.6  & 8.6   &        & -    & -    & -   
\end{tabular}
\caption[caption]{network routing statistics \\ $*$ includes only pairs connected by all networks.}
\label{table:travel_time_stats}
\end{table}

\begin{table}
\centering
\caption{travel time statistics}
\label{table:travel_time_stats}
\end{table}

\subsection{Changes in Routing}


\subsubsection{compare route across directedness}
\begin{figure}
\centering
\caption{example of routing on directed network 1}
\label{fig:routing_1}
\end{figure}

\begin{figure}
\centering
\caption{example of routing on undirected network 1}
\label{fig:routing_1}
\end{figure}

\subsubsection{compare route across level 1 and 2}

Seen in figure XXXX  compare longest path for directed network 2 to the path between those nodes in directed network 1. 

Compare longest path for directed network 1 to undirected network 1. 

compare o/d pair with largest increase in distance between directed 1 and directed 2 to the distance in undirected 2. 

Is allowing travel in any direction on side roads a good replacement for primary and trunk routes?


\begin{figure}
\centering
\caption{example of routing on directed network 1}
\label{fig:routing_1}
\end{figure}

\begin{figure}
\centering
\caption{example of routing on directed network 2}
\label{fig:routing_1}
\end{figure}

\subsubsection{compare across level 2 directedness}

\begin{figure}
\centering
\caption{example of routing on directed network 2}
\label{fig:routing_1}
\end{figure}

\begin{figure}
\centering
\caption{example of routing on undirected network 2}
\label{fig:routing_1}
\end{figure}


\subsection{Accessibility}

in qgis, color lsoa's by travel time to central lsoa. 

in QGIS color lsoa's by travel time from lsoa with highest average distance. 

in QGIS color lsoa's by  average directness, distance divided by straightline distance. 

Plot a few paths in osmnx to look at low directness. 


\begin{figure}
\centering
\caption{lsoas colored by directness of routes to other lsoas}
\label{fig:lsoa_directness}
\end{figure}



\subsection{Notes about computation}

runtimes for the distances between nodes were long. computations were done on an intel i74700HQ processor with the database contained on the internal Solid State Drive. 

Runtimes increased with the number of connected origin destination pairs, since unconnected pairs  did not require the calculation of a shortest path. Thus bike 1 travel times took longer than bike level 2. Additionally, the undirected network calculations took substantially longer than the directed networks because there were sugnificantly more route possibilities with more edges available at each node. 

Table XXXX contains calculation times by network type. 

% table for computation times across algorithm types

% table for computation times across network types
%\begin{table}[]
%\centering
%\begin{tabular}{lllll}
%network  & 1 & 1 undirected & 2  & 2 undirected   \\
%time     & 24:00   & 72:00??  & 9:03  & 36:00        \\        
%\end{tabular}
%\caption{Computation Times}
%\label{table:1}
%\end{table}

\begin{table}[]
\centering
\begin{tabular}{@{}l|llll@{}}
network     & 1           & 1 undirected & 2    & 2 undirected \\ \hline
time(hours) & $\sim$24:00 & $\sim$72:00  & 9:03 & $\sim$36:00 
\end{tabular}
\caption{Calculation times for routes}
\label{table:net_calc_times}
\end{table}

\begin{table}
\centering
\caption{computation times using different algorithms}
\label{table:comp_times_algo}
\end{table}


text


\section{Conclusions}

	\subsection{Results}

\subsection{Limitations}

text

\subsection{Opportunities for improvement and extension}

text


\pagebreak
\section{Appendix}


\pagebreak

\printbibliography

\end{document}
