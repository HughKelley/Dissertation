% 158 words

Accurate representations of street networks from the point of view of an urban cyclist should account for the level of stress that the cyclist experiences when traveling on the network. When streets with a level of stress that exceed a cyclists tolerance are removed from the network, some journeys become impossible or unacceptably long. To increase cycling's share of trips in a city, building a network of infrastructure that connects destinations without exceeding a certain stress tolerance is a key priority. This dissertation implements this methodology using data from OpenStreetMap (OSM), a community generated spatial database. It finds that while difficult, it may be possible to rely on the OSM classification of streets and tagging of cycle infrastructure where professionally collected survey data is not available. Streets tagged ``Primary'' and ``Trunk'' are nearly irreplaceable in the London street network. Once these streets are removed over 40\% of journeys become impossible and the length of the remaining routes increase by approximately 30\%. 