% word count
% 621


%\subsection{Key Questions}
%what was the scope and how was it defined \\ 
%what filters were used \\
%what were the walking and cycling speeds used \\
%what were the travel times calculated \\
%what was the directness of travel on different networks \\
%what is the distribution of travel times in the different networks \\
%what lsoas changed the most and least across networks \\
%how did removing road types affect the number of lsoa centers connected? \\
%how long did the calculations take? \\
%Did removing directionality from the networks have a significant effect? \\
%Did removing directionality mediate the effect of removing primary and trunk streets? \\
%


\subsection{Results}	

The primary conclusion, consistent with observation around London is that cycling on unprotected parts of high volume, main streets is a required part of moving around the city by bicycle. A high percentage of trips require the use of these types of highways. 

A reasonable representation of the London cycling network could be constructed from Open Street Map data using \texttt{OSMnx} and \texttt{Networkx}. With some specific improvements to the software and data a good job could be done of estimating accessibility in London by bicycle. Access to computing resources is a key consideration.

There were significant regional differences in the effect of removing primary and trunk streets and those looking to improve the London cycling network may want to focus their efforts on western and northern London within the scope of the analysis. Removing direction restrictions for cyclists is probably not a good way to improve cycling safety and the directness of travel by bicycle. This did not mediate much of the change in travel times from removing primary and trunk streets. 

Compared to the conclusions of \textcite{furth2016networks}, the London network did not fracture into subcomponents as the San Jose street grid did in that analysis. In the tree-like street network of London decreases in connectivity came through the complete separation of single nodes from all other nodes so that most nodes that lost connectivity with other nodes on the network were likely to lose connectivity with all other nodes. 

\subsection{Limitations}

Selection of origin and destination locations was an important determinant of the results. QUANT uses the transit station of highest degree as the origin and destination points, meaning that there is no walking time included in the calculation. Additionally, QUANT travel times are probabilistic in the sense that the precise start time affects the travel time because there may be a wait time for a bus or train. This is a key advantage that cycling has over public transport, a more linear relationship between distance and travel time. In one instance, the selection of a node at the end of a one way service street meant that only on the undirected networks was it connected to any other. Some LSOA's that were disconnected from the network after the removal of edges continued to contain nodes that were connected to the network. Thus the selection criteria could have a significant impact on the outcomes and could possibly be improved. 

The network representations could also be improved. OSMNX makes working with the Overpass API easier by abstracting away a lot of the process. Unfortunately this ease comes with a cost, a diminished ability to build an accurate network using all of the possible attributes of OSM geometries. A key feature for OSMnx that would improve the ability to define networks accurately would be the option to ``stack'' multiple types of requests into a full network. Collecting several sets of edges and nodes with several filters, as can be accomplished with the Overpass Tubo OSM tool \parencite{overpass_turbo}. 


It was subsequently found that Google speed estimates for cycling can vary pretty dramatically and the 13 kph speed used may be fairly conservative with subsequent tests of Google Maps speed estimates being closer to 15 kph. This does not change the overall indication of the data that public transit is faster than cycling. An empirical study of observed cycling speeds would be the optimal way to settle this question. 

\subsection{Opportunities for Improvement and Extension}

Beyond improving the implementation of the methodology used in this dissertation, the key improvement to the methodology generally would be to move from using OSM tags for highway importance, which is a discrete set of values, to a continuous estimation of the stress for a given street based on all of the possible data. As found in the section \ref{literature} review of literature, there are not clear methods for this estimation problem because of the nature of the data and difficulty of measuring cyclist volumes at a high resolution. To date no methodology has implemented a continuous estimation although \textcite{boisjoly2019bicycle} approach a basic version of this by measuring the portion of a trip completed on designated cycle infrastructure. 

while removing edges from the network did have a significant effect on connectivity, reducing the connected origin and destination pairs by 40\% the study showed that less confident cyclists have route options available to them if they are willing to go out of their way along less direct paths.

Several data sets could potentially augment this type of analysis further. London has several decades of traffic incident data with spatial data associated with an incident. This could possibly provide for validation of the high stress nature of OSM highways tagged primary and trunk. With the beginning of collection on cyclist volumes in some parts of the city, this could become possible in the future.  Additionally, high resolution journey to work data for London exists but is not publicly available due to privacy concerns. Access to this data would allow for a connectivity ratio like that of \textcite{furth2016network} that adjusts the percentage of possible trips of a reasonable length by the number of commuter trips that are associated with that origin and destination. 


This raises the topic of "rat-running" in London, where traffic filters off of main routes onto side roads to avoid traffic. This behavior is a key reason that Open Street Map highway types may not have a real relationship with traffic stress or danger. A "quiet" street may get relatively more traffic than it is built to handle relative to primary routes. this is consistent with anecdotal observations of the author that on some quiet ways traffic behavior is more dangerous and less predictable than on primary routes . it is not clear that dense slow moving traffic is more dangerous than sparse high speed traffic on back roads. 

This would argue for making quiet ways through streets for only cyclists using "gates" .


Methodology is binary, edges are either included or excluded. What would be better is to have a trip ``risk budget''  or allow a cyclist to walk their bike safely through a dangerous intersection. 
