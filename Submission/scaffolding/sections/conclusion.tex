\subsection{Results}

Something here about spanning trees not bein very resiient? 

Most generally, it was found that a decent representation of the London cycling network could be constructed from Open Street Map data using OSMnx and Networkx. With some specific improvements to the software and data a good job could be done of estimating accessibility in London by bicycle. Access to comuting resources is a key consideration.

\subsection{Limitations}

text

\subsection{Opportunities for improvement and extension}


One key way that planners could mediate the effects of competition for space on primary roads is by removing direction restrictions on side roads. 

the variability of cycling travel times was lower than public transport, but public transport allowed efficient access to the highest number of employment opportunities. 


Public transport accessibility was more strongly linked with income reflecting cyclings potential to mediate income inequality by leveling the transportation playing field

while removing edges from the network did have a significant effect on connectivity, reducing the connected origin and destination pairs by XXXXXX\% the study showed that less cconfident cyclists have route options available to them if they are willing to go out of their way along less direct paths. 


This raises the topic of "rat-running" in London, where traffic filters off of main routes onto side roads to avoid traffic. This behavior is a key reason that Open Street Map highway types may not have a real relationship with traffic stress or danger. A "quiet" street may get relatively more traffic than it is built to handle relative to primary routes. this is consistent with anecdotal observations of the author that on some quiet ways trafffic behavior is more dangerous and less predictable than on primary routes . it is not clear that dense slow moving traffic is more dangerous than sparse high speed traffic on back roads. 

This would argue for making quiet ways through streets for only cyclists using "gates" .