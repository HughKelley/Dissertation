%777 words


\begin{quote}
``When you get these incomplete networks you have this situation where great, you have this fresh new bike lane, you're excited you know?" Maerowitz said. "This infrastructure is working for you and then suddenly it's gone and you have to go back out into traffic again.'' - \cite{juhasz2019}.
\end{quote}
 
\subsection{The Challenge of Cycling in Cities}

What makes a city comfortable for cycling? The conditions are obvious when one sees them, but defining the characteristics quantitatively is challenging. More than half of the world's population lives in cities. Cycling is of obvious importance to the well-being of the world's  urban population in terms of the macro-climate crisis, local pollution problems,  public health, wealth disparities, and urban traffic congestion. A commonly researched question is ``what factors have the most significant effect on cycling rates?'' but only recently has cycling been addressed from a network perspective. A comprehensive understanding of cycling networks is hindered by multiple challenges. One is access to and the existence of necessary data. Another is the availability of tools for constructing data into a representational network and assessing that that network, which are only beginning to emerge.   

First is the scope of the analysis, considering individual changes to roads instead of the status of the entire transportation network from the cyclist's perspective.  The majority of work on this question approaches the matter from a perspective of discrete policy intervention and the effect of marginal improvements on cycling. This fails to build a comprehensive theory of cycling rates that a network analysis approach can offer. Only network approach can offer a complete understanding of the level of service for cyclists in a community. 

The second consideration is the difficulty of obtaining data used by the studies that do take a comprehensive network approach. 

%Basic attitude is that improvements to infrastructure only matter to the extent that they improve safe connectivity to ``important'' nodes/edges.  The implication of this view is that infrastructure changes are unlikely to have a meaningful effect on behavior in the system until a critical mass or tipping point is reached in the network. 

In this context, this dissertation contributes to the understanding of cycling infrastructure and cycling behavior by using London as a case study. London is an attractive case because it has a considerable cycling population but does not yet have the level of infrastructure that exists in world leading cities like Copenhagen \parencite{mayor}. Thus London's position at a mid-point of cycling infrastructure development allows strengths and weaknesses to a live program of improvement to be identified.


%%%%%%%%%%%%%%%%%%%%%%%%%%%%%%%%%%%%%%%%%%%%%%%%%%%%%%%%%%%%%%%%%%%%%%%%%%%%%%%%%%%%%%%%%%%%%%%%
\subsection{Research Question}
%
%\subsection{Key Questions}
%what was the scope and how was it defined \\
%How well does Open Street Map serve as a data source \\
%what filters/networks were used \\
%what are reasonable walking and cycling speeds \\
%what were the travel times calculated \\
%what was the directness of travel on different networks \\
%what is the distribution of travel times in the different networks \\
%How does transit by public transport compare to cycling? \\
%what lsoas changed the most and least across networks \\
%how did removing road types affect the number of lsoa centers connected? \\
%how long did the calculations take? \\
%Did removing directionality mediate the effect of removing streets? \\


The goal of this dissertation is to build and interpret quantitative measures for the quality of London's cycle network. First, OpenStreetMap data will be assessed as the primary data source for building representations of London's cycle infrastructure. Second, metrics will be calculated and compared for the different network representations. Third the comparisons will be used to draw conclusions about the London cycling infrastructure network. The most important objective of this dissertation is to specify how data and methods can be combined and improved to provide for a fuller understanding of the current quality of the networks and to indicate the best ways to improve the network in the future. 

The dissertation will address a few secondary questions. First, to what extent is transportation space in London a zero sum game? Does improving the experience of cyclists require taking space from motorists? Second, are there any key differences between the grid-like networks of past research in North American Cities and the more tree-like network of London streets? 

%This research intends to define the overall suitability of a city to cycling using estimates of roads cyclist are and are not willing to use and the effect these decision have on travel times. The central research question is: \textit{how does a preference for safety decrease accessibility in central London when traveling by bicycle?} It extends the literature in three ways; extending quantitative research to a holistic approach rather than focusing on individual edges or nodes, extending holistic approaches to quantitative outcomes rather than visual or anecdotal conclusions, and by emphasizing the use of open source tools and data that are available to any member of community. 
%
%The research provides a methodology that allows a community to identify nodes and edges for improvement that will have the largest impact on the overall network, quantify the expected improvement, and thus require from elected officials specific changes to the roads in their communities.  
%
%The issue implied by these research questions is: to what extent is street level transportation space a zero-sum game? 
%
%Can a high level of service for cyclists be built without taking large amounts of important space away from automobile traffic? 
%
%The intended outcomes of this research include(1) an understanding of the roles of location and connectivity for cycling infrastructure and (2) an understanding of the usefulness of Open Street Map data for estimating these sorts of measures. 

\subsection{Ethical Risks}

This project relies entirely on publicly accessible data. For this reason, ethical risks are not present in the research methodology.  

\subsection{Research Structure}

First, existing work on this research area will be reviewed and the techniques to be built upon will be identified in section \ref{literature}.  Then  section \ref{methods} will describe a general methodology for defining the strength of cycling infrastructure in a city. Section \ref{data} will describe the data available in the context of past work, and what is available for other cities and through other channels that were not available to this research. The steps taken for data cleaning, transformation and joining will be specified. Section \ref{analysis} will describe the implementation of this methodology for London, identifying the scope of the case study, defining the exact data collected and transformed and the specific tools used, reporting the results of the implementation. Finally conclusions will be drawn in section \ref{conclusions}, areas of further research specified, opportunities to improve the methodology and the quality of the data emphasized and the key recommendations for further improving the London cycling infrastructure network will be made. 

After critically reviewing this existing research, a methodology for using open source tools and data to estimate the relative strength of a cycle network and a method for prioritizing improvements will be specified.

The review that follows takes the structure of an initial look at typical literature considering how to promote cycling, recent attempts to use network analysis to accomplish this same goal, and closes with a look at work analyzing data sources relevant to this investigation. 

%\subsection{The role of the researcher}
%
%Read other versions of this and see if it makes sense for this 


