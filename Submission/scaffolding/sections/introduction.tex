
Cycling in cities generally

Cycling in London

safety


``When you get these incomplete networks you have this situation where great, you have this fresh new bike lane, you're excited you know?" Maerowitz said. "This infrastructure is working for you and then suddenly it's gone and you have to go back out into traffic again.'' \cite{juhasz2019}. 

need for open data and methods for people to advocate in places where local government is not supportive

More than half of the world's population lives in cities. Cycling is of obvious importance to the well being of the world's  urban population in terms of the macro-climate crisis, local pollution problems,  public health, wealth disparities, and urban traffic congestion. A commonly researched question is ``what factors have the most significant effect on cycling rates?'' The research has only recently begun to address cycling from a network perspective. 

A large body of research addresses the factors influencing the decision to cycle in an urban environment, and which factors make cycling in that environment more or less safe. Modern quantitative methods have only begun to be applied to this problem though. Challenges come through the lack of technical knowledge among interested parties, lack of access to tools for analysis of this type of complex spatial problems, and lack of access to the data necessary to define the nature of the problem and estimate the outcomes of possible solutions. 

 The review of literature that follows will argue that the majority of work on this question approaches the matter from a perspective of discrete policy intervention and the effect of marginal improvements on cycling. This fails to build a comprehensive theory of cycling rates that a network analysis approach can offer. Only network theory can offer a high level look at the level of service in a community. 

Thus a comprehensive approach to encouraging cycling in an urban area would consider individual improvements in the context of the existing network, and how the improvements would change the size of the low-stress nework coverage of the city. 

This literature review will show that existing research is limited by two considerations. First is the scope of the analysis, considering individual changes instead of the status of the entire transportation network from the cyclists perspective.  Second is the difficulty of obtaining data used by the studies that do take a comprehensive network approach. 

After critically reviewing this existing research, a methodology for using open source tools and data to estimate the relative strength of a cycle network and a method for prioritizing improvements will be specified. 

The review that follows takes the structure of an initial look at typical literature considering how to promote cycling, recent attempts to use network analysis to accomplish this same goal, and closes with a look at work in network analysis that can inform applying the field to cycle networks.

 The implication of this view is that infrastructure changes are unlikely to have a meaningful effect on behavior in the system until a critical mass or tipping point is reached in the network. 

Basic attitude is that improvements to infrastructure only matter to the extent that they improve safe connectivity to ``important'' nodes/edges. 

In this context, this work hopes to make a contribution using London as a case study. London is an attractive case because it has a considerable cycling population but does not yet have the level of infrastructure that exists in world leading cities like Copenhagen. Thus London's position at a mid point of cycling infrastructure development allows strengths and weaknesses to a live program of improvement to be identified. 

%%%%%%%%%%%%%%%%%%%%%%%%%%%%%%%%%%%%%%%%%%%%%%%%%%%%%%%%%%%%%%%%%%%%%%%%%%%%%%%%%%%%%%%%%%%%%%%%
\subsection{Research Goal and Overview}

What makes a city comfortable for cycling? The conditions are clear when one sees them, but defining the characteristics quantitatively is challenging. 

This research intends to define the overall suitability of a city to cycling using estimates of the roads cyclist are and are not willing to use and the effect these decision have on travel times. The central research question is: \textit{how does a preference for safety decrease accessibility in central London when traveling by bicycle?} It extends the literature in three ways; extending quantitative research to a holistic approach rather than focusing on individual edges or nodes, extending holistic approaches to quantitative outcomes rather than visual or anecdotal conclusions, and by emphasizing the use of open source tools and data that are available to any member of community. 

The research provides a methodology that allows a community to identify nodes and edges for improvement that will have the largest impact on the overall network, quantify the expected improvement, and thus require from elected officials specific changes to the roads in their communities. 

The subquestions addressed in the methodology are, What is the quality of the Open Street Map metadata for streets in London? How important are the OSM highway types to efficient transit within London? How does transit by bicycle compare to transit by public transport within London? 

The issue implied by these research questions is: to what extent is street level transportatoin space a zero sum game? Can a high level of service for cyclists be built without taking large amounts of important space away from automobile trafic? 

The intended outcomes of this research include(1) an understanding of the roles of location and connectivity for cycling infrastructure and (2) an understanding of the usefulness of Open Street Map data for estimating these sorts of measures. 


\subsection{ethical risks}

This project relies entirely on publicly accessible data. For this reason, ethical risks are not present in the research methodology.  



\subsection{Research Structure}

First, existing work on this research area will be reviewed and the techniques to be built upon will be identified.  Then a methodology for defining the strength of cycling infrastructure in a city will be specified. The data available for analysis will be described in the context of past work, and what is available for other cities and through other channels that were not available to this research. the steps taken for data cleaning, transformation and joining will be specified. Section 3 will describe the implementation of this methodology for London, identifying the scope of the case study, defining the exact data collected and transformed and the specific tools used. 

The multiple stages of results will be reported, and interpreted. Finally conclusions will be drawn, areas of further research specified, opportunities to improve the methodology and the quality of the data emphasized and the key recommendations for further improving the London cycling infrastructure network will be made. 


\subsection{The role of the researcher}

Read other versions of this and see if it makes sense for this 



\subsection{research question}

\subsection{ethical risks}

\subsection{Research Structure}

\subsection{The role of the researcher}

