% 2425 words

\subsection{Cycling Behavior}

The most important contribution from literature that tries to predict cycling behavior is the focus on multiple types of cyclists, and the factors that influence each type's decision to cycle. Most often, researchers identify four types of cyclist, ``strong and fearless'', ``enthused and confident'', ``interested but concerned'', and ``no way no how'' \parencite{dill2013four}. This categorization is sometimes changed so that the final category is dedicated to children with their set of safety requirements for a suitable cycling environment \parencite{mekuria2012low}. The literature separates the decision to cycle into how often someone who is willing to cycle generally chooses to do so (cycling frequency), and the decision to cycle at all, with different factors influencing each decision \parencite{stinson2005comparison}. 

Despite behavioral differences between these categories of cyclists, research has shown that all cyclists are willing to sacrifice time and energy for increased safety on their route (\cite{winters2011motivators}). Indeed, psychological research showed that fear is a significant factor during an urban cycling trip (\cite{ellett2018state}). This is important given the sensitivity cyclists show to efficiency (\cite{wuerzer2015cycling}). The impact of a route change can be very significant; for instance a higher frequency of stop signs on a route can double the energy required for a journey (\cite{fajans2001bicyclists}). Given the common trade-off between efficiency and safety, it was found that the effect of infrastructure improvements is very dependent on context. Effect is a function of the change in safety, and the importance of the location to trips (\cite{kondo2018bike}). Improvements that meaningfully increase safety at important locations have the largest effect. 

Several studies looked at the importance of perception in behavior change, assuming that a real change in safety is irrelevant if it is not perceived by potential cyclists as a change \parencite{li2012physical} and \parencite{parkin2007models}. This gives rise to literature that focuses on a behavior and attitude change approach from psychology that prioritizes change in habits and perception over infrastructure, with changes to the built environment only used where required to change perception \parencite{savan2017integrated}. This should be considered in the context of research showing that cyclist perceptions of danger are generally accurate \parencite{vandenbulcke2014predicting}. Thus it seems reasonable to conclude that although the decision to cycle is a fairly complex mix of factors, when an urban commuter is deciding whether to cycle, the most important factors are safety, and efficiency. An efficient network of safe street-edges, connecting important place-nodes, would be expected to have a meaningful effect on the rate of cycling in an urban area.  

%%%%%%%%%%%%%%%%%%%%%%%%%%%%%%%%%
\subsection{Cycling Networks}

\textcite{buehler2016bikeway} is a very useful introduction to the literature on cycling networks. Unfortunately it concludes that very little true network analysis has been developed for cycle networks. They found that the majority of papers could be categorized into those that focus exclusively on nodes, and those that focus on edges of the dual graph, where intersections are nodes and streets are edges. At the time of writing there were 115 papers citing Buehler and Dill's review; however all but five of them fail to take the central recommendation that \textit{If individual characteristics of a network's links and nodes contribute to cycling levels, it logically follows that a network of such features would as well...}. The ``Toward Studying the Whole Bicycling Network'' section of Buehler and Dill's review is a good overview of attempts up to 2016. The key findings were that continuity and connectivity of infrastructure are valued by cyclists. Of particular interest is the \textcite{schoner2014missing} study of the relationship between network characteristics and cycling mode share in 74 US cities, finding that the density of the network had the highest elasticity of effect on cycling rate. 

Several of the works reviewed contribute new ways of measuring ``quality'' of the infrastructure. These quality measures include a Bicycle Compatibility Index (BCI) \parencite{klobucar2007network},  Bicycle Level of Service (BLOS) \parencite{lowry2012assessment}, and Level of Traffic Stress (LTS) \parencite{mekuria2012low}. Each of these can reasonably be viewed as an attempt to measure the ``safety'' of a network link. These studies generally lacked a rigorous method for prioritizing nodes by importance or defining a sample set of trips between nodes. Improvements in this area will be addressed in the section reviewing network analysis literature. 

Buehler and Dill note that a key challenge to using the network methods reviewed is data availability, this dissertation hopes that network analysis can be a method for reducing rather than extending the amount of data necessary to understand a cycle network, as network statistics could be used to replace some empirical measurements as discussed below.  They further criticize the approaches as lacking empirical validation. Gathering accurate cycle traffic data and safety data is an immense challenge as demonstrated by the flow estimation techniques of \textcite{gosse2014estimating} and the safety estimate techniques of \textcite{puchades2018role}. The latter focuses on near misses as a proxy for predicting actual safety incidents, but acknowledges the difficulty of collecting near miss data without human observation. 

Since the publication of Buehler's review, the papers extending the full network analysis method have had moderate success. \textcite{akbarzadeh2018designing} use taxi trip data to weight the links between destinations in order to build communities of nodes that tend to be origin and destination pairs. While this is a novel approach to prioritizing edges, it seems likely that taxis trips tend to be used for one time trips which could be very different from daily journey to work trips \parencite{jtw}.

\textcite{doorley2019designing} focus on building cycle infrastructure to maximize a function of travel costs, infrastructure costs, health, traffic accidents, and pollution. While this is an interesting approach, it addresses a more political question in the sense that the key result of the algorithm is to recommend a specific amount of investment in cycle infrastructure to maximize the costs and benefits to all road users, The authors fail to recognize the prioritizing the goals of public policy is a normative and subjective exercise and that a model designed to give an ``objective'' answer to this question inherently reflects the author's preferences and when calibrated to ``the real world'' reflects the biases and preferences of the status quo, rather than the true ideal outcomes preferred by a population. Instead, modeling, especially for urban planning purposes, should accept an exogenous goal and implement it as efficiently as possible. For instance, cycle infrastructure, is explicitly intended to reduce motor vehicle use, it would make no sense to then use a model that determines the efficient level of motor vehicle use, the political process has already determined the answer and merely asks for implementation recommendations from the modeler. 

\textcite{mauttone2017bicycle} similarly focuses on an optimization framework for cycling networks, choosing a subset of streets that are ``suitable to building cycle infrastructure''. This is odd in the sense that the goal of building cycling infrastructure is to \textit{create} streets that are suitable for cycling, not merely identify them. Similar to \textcite{doorley2019designing}, they identify a cost to building cycle paths which they seek to balance against the benefits. 

Overall, it is not clear that a model for building cycle paths should be particularly cost sensitive. \textcite{gu2017cost} found a very high return on investment to the budget for cycling infrastructure in New York City. The very idea of using network analysis for the development of cycle networks emphasizes the potential non linearity of the effect of building more infrastructure, with usage accelerating as the network approaches ``completeness'' in some form. In addition, cities tend to combine cycling infrastructure improvement with other required improvement and maintenance activities, mitigating the costs by being opportunistic in implementation. 

Lastly, \textcite{osama2017models} use a number of predictors including network statistics to predict bike travel within zones of Vancouver similar to \textcite{schoner2014missing}. They found a positive coefficient for the density of the bike network in a zone. fThus while network analysis has been applied to cycling infrastructure there is not a consensus on the methodology to be used. Additionally a definition of ``quality'' or ``safety'' has not been clearly established. 

The most direct inspiration for this work comes from \textcite{furth2016network}. That analysis built representations of the San Jose cycle network for different types of users employing data collected and maintained by the local government. Data used included the locations of cycle tracks and shared paths, the width of street lanes, the width of bike lanes on those streets,  the volume of traffic by lane for each street, right of way in intersections and the structure of each intersection, and the frequency of bike lane blockages for each street. The analysis then calculated from the data on journeys to work a ``connectivity ratio'' that is the percentage of commuter trips that are possible for a given representation of the network. 

Furth notes that ``These connectivity methods do not necessarily require use of the LTS classification scheme. They can be applied with any classification scheme that distinguishes high- and low-stress segments.'' This is a vital component of this dissertation as the majority of data that Furth uses is not available for London. In the methodology section, a process for building similar networks using data from OpenStreetMap will be specified. Furth also notes the computational burden of the analysis as a limiting factor. 

The second important inspiration for this dissertation is \textcite{boisjoly2019bicycle}. This work used a survey of cyclist route choices between sets of destinations in Montreal to estimate the probable path for a larger set of destinations. With these paths, they estimated the average directness of a journey in the city to identify neighborhoods with relatively low directness of cycling routes.  Their analysis considered for a given route the percentage of the routes distance that occurred on cycling infrastructure and the directness of the route relative to the directness of the shortest possible route between the given origin and destination. It looked particularly at the idea that ``there is often a trade-off between route directness and quality of route'' \parencite{boisjoly2019bicycle}.  The routes were predicted using data from a survey of 1,525 cyclists, which collected data on their cycle trips regarding usage of bicycling infrastructure. The study specified a cost function that allowed for the expression of stress and distance in common units by penalizing high stress edges, increasing their distance cost, and reducing the distance cost of low stress edges. The cost function was estimated from the survey of cyclists. It measures how far from the shortest route the cyclist diverted in order to use a piece of cycling infrastructure. 

The strength of the Boisjoly study is that it does not require high detail data on the street network, only the simple street connectivity network and data on the location of cycling infrastructure. The weakness of the study is twofold. It requires substantial survey data on route choice from network users and it uses minimal discrimination across the quality of cycle infrastructure, and across the level of stress for a given street. 

%%%%%%%%%%%%%%%%%%%%%%%%%%%%%%%%%%%%%%%%%%
\subsection{OpenStreetMap Data Quality}\label{OSMQuality}

A central research question of this work is "how sufficient is Open Street Map (OSM) data for replicating the studies considered above." This is because both the Furth and Boisjoly studies rely on data unavailable for London, cycling route choice survey data and high detail street characteristic data. 

There is a strong body of existing research on the quality of OSM data. This research can be divided into two focus areas and two types of methodology. The two focus areas are locational accuracy of features and the completeness of attribute tagging and description for features. The two methods for assessing the quality of the data are extrinsic and intrinsic. The extrinsic method uses an external professionally collected dataset. The intrinsic analysis attempts to solely use an analysis of OSM data to assess itself. 

Studies on locational accuracy are consistently extrinsic. \textcite{haklay2010good} compared OSM in the UK to the government produced Ordnance Survey data for roads in the UK finding that there was about 24\% coverage of the UK and that features tended to be very close to their location in the Ordnance survey data. This is of minimal use to assessing cycle networks however, because the characteristics of the features, streets, are of much more importance than the precision of their locations. 

Assessing the accuracy and completeness of feature attribute tagging in OSM is a more recent endeavor and more difficult due to the lower availability of data, the more frequently changing nature of the data as road works are undertaken and the very open structure of tagging of attributes in OSM.  

The use of OSM data for specialized routing applications has been considered by \textcite{mobasheri2017crowdsourced}. In this study, they considered the quality of sidewalk/pavement tagging in OSM from multiple cities for the purpose of routing wheelchair users. This study was looking for information about the surface type, incline, and width of pedestrian ways. They found about 17\% coverage of sidewalks in Hamburg Germany and that coverage was best where density of features and tags was highest. This work combined extrinsic and intrinsic analysis of the question but did not take the opportunity to validate the intrinsic analysis with the extrinsic analysis. They concluded that large parts of the cities considered had OSM data that could support specialized routing if the data quality was confirmed by additional checks. 

\textcite{hochmair2013assessing} researched the completeness of bicycle features in OSM. They used Google Maps to extrinsically validate the OSM data for bike trails in the US and Europe. They found that coverage was fairly high, but they only considered fully separated bicycle infrastructure like segregated lanes and off-road trails, which is of limited use to the analysis of urban cycling networks where many cycling infrastructure features exist on shared roads. 

Finally, \textcite{zielstra2013assessing} considered the impact of bulk uploads of geospatial data to OSM. They found that in the United States, while government collected data had a higher level of completeness for motor vehicle related street network data, data for pedestrian related features was higher in OSM. This raises the possibility that for a well mapped area, OSM could be the best possible source of data for cycling, depending on the priorities of local governments in their data collection efforts. This is especially important in the context of the new Transport for London Cycling Infrastructure Database, which will be imported to OSM over the next few years \parencite{tflcid}. Despite the possibility that there are meaningful problems with current cycle related data in OSM today, OSM is likely to be the highest quality source of this data in the near future. 

Ultimately, the quality of OSM data for cycling related data is uncertain and the methodology and analysis will address a qualitative attempt to understand accuracy in London, with the expectation that it is at least as high as anywhere else in the world. 
